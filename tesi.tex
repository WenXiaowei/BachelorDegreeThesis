% I seguenti commenti speciali impostano:
% 1. 
% 2. PDFLaTeX come motore di composizione;
% 3. tesi.tex come documento principale;
% 4. il controllo ortografico italiano per l'editor.

% !TEX encoding = UTF-8
% !TEX TS-program = pdflatex
% !TEX root = tesi.tex
% !TEX spellcheck = it-IT

\documentclass[11pt,                    % corpo del font principale
a4paper,                 % carta A4
twoside,                 % impagina per fronte-retro
openright,               % inizio capitoli a destra
english,
italian,
ctexart,
]{book}

%**************************************************************
% Importazione package
%************************************************************** 
%\usepackage{amsmath,amssymb,amsthm}    % matematica
\usepackage[UTF8]{ctex}
\usepackage[T1]{fontenc}                % codifica dei font:
% NOTA BENE! richiede una distribuzione *completa* di LaTeX

\usepackage[utf8]{inputenc}             % codifica di input; anche [latin1] va bene
% NOTA BENE! va accordata con le preferenze dell'editor

\usepackage[english, italian]{babel}    % per scrivere in italiano e in inglese;
% l'ultima lingua (l'italiano) risulta predefinita

\usepackage{bookmark}                   % segnalibri

\usepackage{caption}                    % didascalie

\usepackage{chngpage,calc}              % centra il frontespizio

\usepackage{csquotes}                   % gestisce automaticamente i caratteri (")

\usepackage{emptypage}                  % pagine vuote senza testatina e piede di pagina

\usepackage{epigraph}            % per epigrafi

\usepackage{eurosym}                    % simbolo dell'euro

%\usepackage{indentfirst}               % rientra il primo paragrafo di ogni sezione

\usepackage{graphicx}                   % immagini

\usepackage{hyperref}                   % collegamenti ipertestuali

\usepackage[binding=5mm]{layaureo}      % margini ottimizzati per l'A4; rilegatura di 5 mm


\usepackage{microtype}                  % microtipografia

\usepackage{mparhack,fixltx2e,relsize}  % finezze tipografiche

\usepackage{nameref}                    % visualizza nome dei riferimenti                                      

\usepackage[font=small]{quoting}        % citazioni

%\usepackage{subfig}                     % sottofigure, sottotabelle

\usepackage[italian]{varioref}          % riferimenti completi della pagina

\usepackage[dvipsnames]{xcolor}         % colori
\usepackage{formattazione}

\usepackage{booktabs}                   % tabelle                                       
\usepackage{tabularx}                   % tabelle di larghezza prefissata                                    
\usepackage{longtable}                  % tabelle su più pagine                                        
\usepackage{ltxtable}                   % tabelle su più pagine e adattabili in larghezza

\usepackage[toc, acronym]{glossaries}   % glossario
% per includerlo nel documento bisogna:
% 1. compilare una prima volta tesi.tex;
% 2. eseguire: makeindex -s tesi.ist -t tesi.glg -o tesi.gls tesi.glo
% 3. eseguire: makeindex -s tesi.ist -t tesi.alg -o tesi.acr tesi.acn
% 4. compilare due volte tesi.tex.

\usepackage[backend=biber,style=numeric-comp,hyperref,backref]{biblatex}
% eccellente pacchetto per la bibliografia;
% produce uno stile di citazione autore-anno;
% lo stile "numeric-comp" produce riferimenti numerici
% per includerlo nel documento bisogna:
% 1. compilare una prima volta tesi.tex;
% 2. eseguire: biber tesi
% 3. compilare ancora tesi.tex.

g

\input{tesi-config}                     % file con le impostazioni personali
\raggedbottom
\begin{document}
%**************************************************************
% Materiale iniziale
%**************************************************************
    \frontmatter
    \input{inizio-fine/frontespizio}
    \input{inizio-fine/colophon}
    % !TEX encoding = UTF-8
% !TEX TS-program = pdflatex
% !TEX root = ../tesi.tex

%**************************************************************
% Dedica
%**************************************************************
\cleardoublepage
\phantomsection
\thispagestyle{empty}
\pdfbookmark{Dedica}{Dedica}

\vspace*{3cm}

\begin{center}
    I wish the day were pure and all I met was tenderness.\\ \medskip
    但愿日子清净, 抬头遇见的都是柔情.\\ \medskip
\end{center}

\medskip

\begin{center}
Dedicato a tutte le persone che mi hanno aiutato per essere qui.
\end{center}

    % !TEX encoding = UTF-8
% !TEX TS-program = pdflatex
% !TEX root = ../tesi.tex

%**************************************************************
% Sommario
%**************************************************************
\cleardoublepage
\phantomsection
\pdfbookmark{Sommario}{Sommario}
\begingroup
\let\clearpage\relax
\let\cleardoublepage\relax
\let\cleardoublepage\relax

\chapter*{Sommario}

Il presente documento descrive il lavoro svolto durante il periodo di stage, della durata di circa trecento ore, dal laureando Xiaowei Wen presso l'azienda Imola Informatica S.p.A. Gli obbiettivi da raggiungere erano molteplici. In primo luogo era richiesto lo sviluppo di un tool in grado di automatizzare la decompilazione di un file \textit{APK} e la ricompilazione del file. In secondo luogo era richiesta l'implementazione di alcune funzionalità di analisi del codice sorgente ottenuto dal passo precedente.
Terzo e ultimo obbiettivo era quello di permettere di eseguire l'app con il proxy in un AVD e di monitorare il traffico di rete generando un file \textit{cap}.

%\vfill
%
%\selectlanguage{english}
%\pdfbookmark{Abstract}{Abstract}
%\chapter*{Abstract}
%
%\selectlanguage{italian}

\endgroup			

\vfill


    \input{inizio-fine/ringraziamenti}
    \input{inizio-fine/indici}
    \cleardoublepage

%**************************************************************
% Materiale principale
%**************************************************************
    \mainmatter
    \input{./capitoli/1_0_introduzione.tex}             % Introduzione
    \input{./capitoli/2_0_descrizione_stage.tex}             % Kick-Off
    % !TEX encoding = UTF-8
% !TEX TS-program = pdflatex
% !TEX root = ../tesi.tex

%**************************************************************
\chapter{Analisi dei requisiti}
\label{ch:analisi-requisiti}
%**************************************************************

\intro{
In questa sezione viene presenta l'analisi dei requisiti del tool, comprensivo dei casi d'uso, requisiti e il tracciamento di questi ultimi.
Per lo studio dei casi di utilizzo del prodotto sono stati creati dei diagrammi.
I diagrammi dei casi d'uso (in inglese \emph{Use Case Diagram}) sono diagrammi di tipo \gls{uml} dedicati alla descrizione delle funzioni o servizi offerti da un sistema, così come sono percepiti e utilizzati dagli attori che interagiscono col sistema stesso.
Essendo il progetto finalizzato alla creazione di un tool per l'automazione di un processo di decompilazione di un file \gls{apk}, di trasformazione dei file decompilati (\gls{dex}) in codice java e dell'analisi di quest'ultimo.
}\\


\section{Attori}\label{sec:attori}
L'unico utilizzatore del software \`{e} identificato come \textbf{attore generico}, ha il permesso di effettuare tutte le operazioni offerte dal tool.
\section{Casi d'uso}\label{sec:casi-d'uso}
Di seguito sono elencati i casi d'uso individuati dallo stagista durante la fase dell'analisi dei requisiti.
\subsection*{UC-1 Selezione file}\label{subsec:uc-1-selezione-file}
\begin{itemize}
    \item \textbf{attori:} utente generico;
    \item \textbf{descrizione:} serve per permettere all'utente di selezionare il file APK da analizzare;
    \item \textbf{pre-condizioni:} l'utente ha avviato il tool;
    \item \textbf{post-condizioni:} l'utente ha selezionato un file di estensione APK valido;
    \item \textbf{flusso degli eventi principali:}
    \begin{itemize}
        \item l'utente seleziona il file;
        \item l'utente conferma la selezione del file.
    \end{itemize}
\end{itemize}
\subsection*{UC-2 Avvio decompilazione}\label{subsec:uc-2-avvio-decompilazione}
\begin{itemize}
    \item \textbf{attori:} utente generico;
    \item \textbf{descrizione:} l'utente deve avviare la decompilazione del file APK;
    \item \textbf{pre-condizioni:} l'utente ha selezionato con successo un file APK;
    \item \textbf{post-condizioni:} la decompilazione è avvenuta con successo;
    \item \textbf{flusso degli eventi principali:}
    \begin{itemize}
        \item l'utente avvia la decompilazione;
        \item l'utente visualizza il messaggio della decompilazione avvenuta con successo;
    \end{itemize}
    \item \textbf{Estensione:}
    \begin{itemize}
        \item UC-3 Visualizzazione errore di decompilazione;
    \end{itemize}
\end{itemize}

\subsection*{UC-3 Visualizzazione errore di decompilazione}\label{subsec:uc-3-visualizzazione-errore-di-decompilazione}
\begin{itemize}
    \item \textbf{attori:} utente generico;
    \item \textbf{descrizione:} la decompilazione del file APK potrebbe generare degli errori;
    \item \textbf{pre-condizioni:} l'utente ha avviato la decompilazione dell'APK;
    \item \textbf{post-condizioni:} l'utente ha visualizzato il messaggio d'errore;
    \item \textbf{flusso degli eventi principali:}
    \begin{itemize}
        \item l'utente visualizza il messaggio di errore;
    \end{itemize}
\end{itemize}
\subsection*{UC-4 Installazione APK decompilato}\label{subsec:uc-4-installazione-apk-decompilato}
\begin{itemize}
    \item \textbf{attori:} utente generico;
    \item \textbf{descrizione:} permette all'utente d'installare l'apk, decompilato, manomesso e ricompilato, su un AVD;
    \item \textbf{pre-condizioni:} la decompilazione dell'APK è stata eseguita con successo;
    \item \textbf{post-condizioni:} l'APK è stato installato sull'AVD con successo;
    \item \textbf{flusso degli eventi principali:}
    \begin{itemize}
        \item l'utente seleziona un'AVD presente sul proprio computer;
        \item l'utente avvia l'installazione dell'APK ricompilato;
        \item l'utente visualizza un messaggio che segnala l'installazione avvenuto con successo.
    \end{itemize}
\end{itemize}
\subsubsection*{UC-4.1 Selezione AVD}\label{subsubsec:uc-4.1-selezione-avd}
\begin{itemize}
    \item \textbf{attori:} utente generico;
    \item \textbf{descrizione:} serve per permettere all'utente di selezionare un'AVD presente nel proprio computer;
    \item \textbf{pre-condizioni:} l'utente ha avviato il software;
    \item \textbf{post-condizioni:} l'utente ha selezionato l'AVD da avviare;
    \item \textbf{flusso degli eventi principali:}
    \begin{itemize}
        \item l'utente visualizza un elenco delle AVD presenti nel proprio computer;
        \item l'utente seleziona un'AVD;
        \item l'utente conferma la selezione;
        \item l'AVD selezionato si è avviato con successo.
    \end{itemize}
    \item \textbf{Estensione}
    \begin{itemize}
        \item UC-5 Visualizzazione messaggio nessun AVD rilevato;
    \end{itemize}
\end{itemize}
\begin{figure}
    \centering
    \includegraphics[width=10cm, height=8cm]{./usecase/uc_4_1.png}
    \caption{Sottocaso d'uso UC-4.1 Selezione AVD}
\end{figure}

\subsection*{UC-5 Visualizzazione messaggio nessun AVD rilevato} \label{subsec:uc-5-visualizzazione-messaggio-nessun-avd-rilevato}
\begin{itemize}
    \item \textbf{attori:} utente generico;
    \item \textbf{descrizione:} quando non sono presenti nessun AVD o il tool non è riuscito a rilevarne, viene mostrato un messaggio all'utente;
    \item \textbf{pre-condizioni:} l'utente ha aperto il tool;
    \item \textbf{post-condizioni:} l'utente ha visualizzato il messaggio;
    \item \textbf{flusso degli eventi principali:}
    \begin{itemize}
        \item l'utente visualizza il messaggio;
    \end{itemize}
\end{itemize}
\subsection*{UC-6 Errore durante avvio dell'AVD}\label{subsec:uc-6-errore-durante-avvio-dell'avd}
\begin{itemize}
    \item \textbf{attori:} utente generico;
    \item \textbf{descrizione:} serve a mostrare all'utente gli eventuali errori durante l'avvio dell'AVD;
    \item \textbf{pre-condizioni:} l'utente ha selezionato un'AVD e ha confermato l'avvio;
    \item \textbf{post-condizioni:} l'utente ha visualizzato il messaggio d'errore;
    \item \textbf{flusso degli eventi principali:}
    \begin{itemize}
        \item l'utente visualizza il messaggio d'errore;
    \end{itemize}
\end{itemize}
\subsection*{UC-7 Dump dello storage interno}\label{subsec:uc-6-dump-dello-storage-interno}
\begin{itemize}
    \item \textbf{attori:} utente generico;
    \item \textbf{descrizione:} nel caso l'utente volesse una copia dei dati interni dell'applicativo, ha bisogno di fare il dump;
    \item \textbf{pre-condizioni:} l'installazione dell'APK manomesso è andata a buon fine;
    \item \textbf{post-condizioni:} l'area di storage dell'app è stata copiata con successo;
    \item \textbf{flusso degli eventi principali:}
    \begin{itemize}
        \item l'utente seleziona la voce "copia i dati interni";
        \item l'utente seleziona il path dove collocare i dati.
    \end{itemize}
\end{itemize}
\subsection*{UC-8 Decodifica del codice}\label{subsec:uc-8-decodifica-del-codice}
\begin{itemize}
    \item \textbf{attori:} utente generico;
    \item \textbf{descrizione:} durante la decompilazione dell'APK vengono creati dei file .dex che contengono il codice sorgente dell'APK, e questo caso d'uso serve per permettere all'utente di ottenere il codice sorgente;
    \item \textbf{pre-condizioni:} la decompilazione è avvenuta con successo;
    \item \textbf{post-condizioni:} l'utente ha ottenuto una copia del codice sorgente in java;
    \item \textbf{flusso degli eventi principali:}
    \begin{itemize}
        \item l'utente ha selezionato la funzionalità decodifica dei dex;
        \item l'utente seleziona il percorso dove posizionare il codice sorgente ottenuto;
        \item l'utente ha salvato il codice sorgente ottenuto.
    \end{itemize}
\end{itemize}
\subsubsection*{UC-8.1 Avvio decodifica}
\begin{itemize}
    \item \textbf{attori:} utente generico;
    \item \textbf{descrizione:} serve all'utente per avviare la decodifica dei file .dex;
    \item \textbf{pre-condizioni:} la decompilazione dell'APK è avvenuta correttamente;
    \item \textbf{post-condizioni:} la decodifica è avvenuta con successo;
    \item \textbf{flusso degli eventi principali:}
    \begin{itemize}
        \item l'utente seleziona la funzionalità di decodifica dei file .dex;
    \end{itemize}
\end{itemize}
\subsubsection*{UC-8.2 Salvataggio del codice decodificato}
\begin{itemize}
    \item \textbf{attori:} utente generico;
    \item \textbf{descrizione:} serve all'utente per salvare i codici sorgenti decodificati;
    \item \textbf{pre-condizioni:} la decompilazione è avvenuta con successo;
    \item \textbf{post-condizioni:} i file con i codici sorgenti sono stati salvati correttamente;
    \item \textbf{flusso degli eventi principali:}
    \begin{itemize}
        \item l'utente seleziona la voce "salva file decodificati";
        \item l'utente seleziona la posizione dove vuole salvare i file;
        \item i file vengono salvati correttamente.
    \end{itemize}
\end{itemize}

\begin{figure}[H]
    \centering
    \includegraphics[width=10cm, height=8cm]{./immagini/usecase/uc_8.png}
    \caption{Caso d'uso UC-8 Decodifica codici .dex}
\end{figure}



\subsection*{UC-9 Visualizzazione errore decodifica}\label{subsec:uc-9-visualizzazione-errore-decodifica}
\begin{itemize}
    \item \textbf{attori:} utente generico;
    \item \textbf{descrizione:} durante la decodifica dei .dex possono sorgere molteplici errori;
    \item \textbf{pre-condizioni:} l'utente ha selezionato la funzionalità di decodifica del codice .dex;
    \item \textbf{post-condizioni:} l'utente ha visualizzato il messaggio di errore durante la decodifica;
    \item \textbf{flusso degli eventi principali:}
    \begin{itemize}
        \item l'utente ha visualizzato il messaggio d'errore;
    \end{itemize}
\end{itemize}

\begin{figure}[H]
    \centering
    \includegraphics[width=10cm, height=8cm]{./immagini/usecase/uc_principali.png}
    \caption{Casi d'uso da 1 a 12}
\end{figure}

\subsection*{UC-10 Firmare l'APK ricompilato}\label{subsec:uc-10-firmare-l'apk-ricompilato}
\begin{itemize}
    \item \textbf{attori:} utente generico;
    \item \textbf{descrizione:} dopo la ricompilazione si può firmare l'APK;
    \item \textbf{pre-condizioni:} la ricompilazione dell'APK è avvenuta correttamente;
    \item \textbf{post-condizioni:} l'APK è stata firmato correttamente;
    \item \textbf{flusso degli eventi principali:}
    \begin{itemize}
        \item UC-10.1 selezione keystore;
        \item UC-10.3 inserimento alias;
        \item UC-10.4 inserimento password.
    \end{itemize}
    \item \textbf{Estensione}
    \begin{itemize}
        \item UC-11 Visualizzazione credenziali keystore errate;
    \end{itemize}
\end{itemize}
\subsubsection*{UC-10.1 Selezione keystore}\label{subsubsec:uc-10.1-selezione-keystore}
\begin{itemize}
    \item \textbf{attori:} utente generico;
    \item \textbf{descrizione:} l'utente deve selezionare il keystore da utilizzare per firmare l'APK
    \item \textbf{pre-condizioni:} la ricompilazione dell'APK è avvenuta correttamente;
    \item \textbf{post-condizioni:} è stato selezionato un file di tipo keystore corretto (estensione: \textbf{.jks});
    \item \textbf{flusso degli eventi principali:}
    \begin{itemize}
        \item l'utente seleziona la funzionalità per selezionare il keystore;
        \item l'utente seleziona il keystore;
    \end{itemize}
    \item \textbf{flussi secondari}
    \begin{itemize}
        \item UC-10.2 visualizzazione messaggio che mostra che il file selezionato non è valido;
    \end{itemize}
\end{itemize}
\subsubsection*{UC-10.2 Visualizzazione messaggio file selezionato non valido}
\begin{itemize}
    \item \textbf{attori:} utente generico;
    \item \textbf{descrizione:} l'utente, al quale è stato chiesto di selezionare un file di tipo keystore, potrebbe selezionare un file non valido;
    \item \textbf{pre-condizioni:} l'utente ha selezionato un file;
    \item \textbf{post-condizioni:} il messaggio di errore è stato mostrato;
    \item \textbf{flusso degli eventi principali:}
    \begin{itemize}
        \item l'utente visualizza il messaggio d'errore.
    \end{itemize}
\end{itemize}
\subsubsection*{UC-10.3 Inserimento Alias}
\begin{itemize}
    \item \textbf{attori:} utente generico;
    \item \textbf{descrizione:} l'utente deve inserire l'alias della chiave da utilizzare durante la firma dell'APK;
    \item \textbf{pre-condizioni:} l'utente ha selezionato un keystore valido;
    \item \textbf{post-condizioni:} l'utente ha inserito l'alias da utilizzare;
    \item \textbf{flusso degli eventi principali:}
    \begin{itemize}
        \item l'utente inserisce l'alias della chiave da utilizzare per firmare l'APK;
    \end{itemize}
\end{itemize}
\subsubsection*{UC-10.4 Inserimento password}
\begin{itemize}
    \item \textbf{attori:} utente generico;
    \item \textbf{descrizione:} l'utente deve inserire la password del keystore da utilizzare durante la firma dell'APK;
    \item \textbf{pre-condizioni:} l'utente ha selezionato un keystore valido;
    \item \textbf{post-condizioni:} l'utente ha inserito la password da utilizzare;
    \item \textbf{flusso degli eventi principali:}
    \begin{itemize}
        \item l'utente inserisce la password;
    \end{itemize}
\end{itemize}
\begin{figure}[H]
    \centering
    \includegraphics[width=10cm, height=8cm]{./immagini/usecase/uc_10.png}
    \caption{Sottocasi d'uso del caso d'uso UC-10}
\end{figure}

\subsection*{UC-11 Visualizzazione credenziali keystore errati}\label{subsec:uc-11-visualizzazione-credenziali-keystore-errati}
\begin{itemize}
    \item \textbf{attori:} utente generico;
    \item \textbf{descrizione:} per firmare l'APK l'utente deve inserire delle credenziali, e quando quest'ultime sono errate viene mostrato un messaggio di errore;
    \item \textbf{pre-condizioni:} l'utente ha selezionato la funzionalità per ricompilare l'APK;
    \item \textbf{post-condizioni:} l'utente ha visualizzato il messaggio di errore;
    \item \textbf{flusso degli eventi principali:}
    \begin{itemize}
        \item l'utente ha visualizzato l'errore;
    \end{itemize}
\end{itemize}

\subsection*{UC-12 Visualizzazione messaggio file non valido}\label{subsec:uc-12-visualizzazione-messaggio-file-non-valido}
\begin{itemize}
    \item \textbf{attori:} utente generico;
    \item \textbf{descrizione:} quando l'utente seleziona un file che non ha l'estensione APK un messaggio deve essere mostrato all'utente;
    \item \textbf{pre-condizioni:} l'utente ha selezionato un file non APK;
    \item \textbf{post-condizioni:} l'utente ha visualizzato il messaggio d'errore;
    \item \textbf{flusso degli eventi principali:}
    \begin{itemize}
        \item l'utente visualizza il messaggio d'errore.
    \end{itemize}
\end{itemize}
\subsection*{UC-13 Decodifica dei file \textit{.dex} in file \textit{.java}}\label{subsec:uc-13-decodifica-dei-filetextitin-filetextit}
\begin{itemize}
    \item \textbf{attori:} utente generico;
    \item \textbf{descrizione:} dall'APK ricompilato si ottengono dei file \textit{.dex} che possono essere convertiti in \textit{.class} e quindi in \textit{.java};
    \item \textbf{pre-condizioni:} la decompilazione dell'APK \`{e} andata a buon fine;
    \item \textbf{post-condizioni:} la decodifica dei file \textit{.dex} in \textit{.java} \`{e} andata a buon fine;
    \item \textbf{flusso degli eventi principali:}
    \begin{itemize}
        \item l'utente seleziona la voce "Decodifica Dex".
    \end{itemize}
\end{itemize}

\subsection*{UC-14 Analisi del codice \textit{.java}}\label{subsec:uc-14-analisi-del-codicetextit}
\begin{itemize}
    \item \textbf{attori:} attore generico;
    \item \textbf{descrizione:} dopo aver ottenuto i file \textit{.java} eseguendo il caso d'uso UC-13, si potr\`{a} effettuare dell'analisi sul codice;
    \item \textbf{pre-condizioni:} la decodifica dei file \textit{.dex} in \textit{.java} \`{e} andata a buon fine;
    \item \textbf{post-condizioni:} viene generato un pdf con i risultati dell'analisi;
    \item \textbf{flusso degli eventi principali:}
    \begin{itemize}
        \item l'utente seleziona la voce "analyze";
        \item l'utente seleziona le opzioni di analisi;
        \item al completamento dell'analisi, l'utente specifica dove posizionare il file pdf con i risultati dell'analisi;
        \item il file viene salvato nel percorso specificato dall'utente.
    \end{itemize}
\end{itemize}
\begin{figure}[H]
    \centering
    \includegraphics[width=10cm, height=8cm]{./immagini/usecase/uc_14_1_14_2.png}
    \caption{Sottocasi d'uso del caso d'uso 14}
\end{figure}



\subsection*{UC-15 avvio AVD}\label{subsec:uc-15-avvio-avd}
\begin{itemize}
    \item \textbf{attori:} attore generico;
    \item \textbf{descrizione:} serve per avviare un'avd per poter eseguire l'applicazione;
    \item \textbf{pre-condizioni:} il tool è stato avviato correttamente e nel sistema è presente almeno un AVD;
    \item \textbf{post-condizioni:} l'AVD è stato avviato correttamente;
    \item \textbf{flusso degli eventi principali:}
    \begin{itemize}
        \item l'attore visualizza l'elenco delle AVD presenti nel sistema;
        \item l'attore seleziona un'AVD che vuole avviare;
        \item l'attore seleziona se utilizzare un proxy da impostare nell'AVD;
        \item l'attore seleziona la voce avvia AVD.
    \end{itemize}
    \item \textbf{estensione:} UC-5 Visualizzazione messaggio nessun AVD rilevato.
\end{itemize}
\subsection*{UC-16 avvio AVD senza proxy}\label{subsec:uc-16-avvio-avd-senza-proxy}
\begin{itemize}
    \item \textbf{attori:} attore generico;
    \item \textbf{descrizione:} quando l'attore decide che non vuole avviare l'AVD modificando le impostazioni di proxy, viene eseguito questo caso d'uso;
    \item \textbf{pre-condizioni:} l'attore ha avviato correttamente il tool e nel sistema è presente almeno un'AVD;
    \item \textbf{post-condizioni:} l'AVD è stato avviato correttamente con le impostazioni di proxy di default;
    \item \textbf{flusso degli eventi principali:}
    \begin{itemize}
        \item l'attore visualizza l'elenco delle AVD presenti nel sistema;
        \item l'attore seleziona un'AVD che vuole avviare;
        \item l'attore seleziona di non utilizzare un proxy da impostare nell'AVD;
        \item l'attore seleziona la voce avvia AVD.
    \end{itemize}
\end{itemize}
\subsection*{UC-17 avvio AVD con proxy}\label{subsec:uc-16-avvio-avd-con-proxy}
\begin{itemize}
    \item \textbf{attori:} attore generico;
    \item \textbf{descrizione:} quando l'attore decide che non vuole avviare l'AVD modificando le impostazioni di proxy, viene eseguito questo caso d'uso;
    \item \textbf{pre-condizioni:} l'attore ha avviato correttamente il tool e nel sistema è presente almeno un'AVD;
    \item \textbf{post-condizioni:} l'AVD è stato avviato correttamente con le impostazioni di proxy inseriti;
    \item \textbf{flusso degli eventi principali:}
    \begin{itemize}
        \item l'attore visualizza l'elenco delle AVD presenti nel sistema;
        \item l'attore seleziona un'AVD che vuole avviare;
        \item l'attore seleziona di utilizzare un proxy da impostare nell'AVD;
        \item l'attore inserisce le informazioni del proxy, per esempio: \textit{localhost:8080};
        \item l'attore seleziona la voce avvia AVD.
    \end{itemize}
\end{itemize}
\subsubsection*{UC-17.1 Selezione voce Avvia con proxy}
\begin{itemize}
    \item \textbf{attori:} attore generico;
    \item \textbf{descrizione:} l'utente vuole avviare l'AVD con le opzioni di proxy;
    \item \textbf{pre-condizioni:} il tool è stato avviato correttamente ed è riuscito a rilevare gli AVD presenti nel sistema;
    \item \textbf{post-condizioni:} l'opzione di proxy è stata selezionata;
    \item \textbf{flusso degli eventi principali:}
    \begin{itemize}
        \item l'utente seleziona la voce per avviare l'AVD.
    \end{itemize}
\end{itemize}
\subsubsection*{UC-17.2 selezione voce avvio AVD}
\begin{itemize}
    \item \textbf{attori:} attore generico;
    \item \textbf{descrizione:} l'utente avvia l'AVD;
    \item \textbf{pre-condizioni:} l'utente ha selezionato le opzioni per avviare l'AVD;
    \item \textbf{post-condizioni:} il tool avvia l'AVD;
    \item \textbf{flusso degli eventi principali:}
    \begin{itemize}
        \item l'utente seleziona la voce avvio l'AVD.
    \end{itemize}
\end{itemize}
\subsubsection*{UC-17.3 inserimento indirizzo e porta proxy}
\begin{itemize}
    \item \textbf{attori:} attore generico;
    \item \textbf{descrizione:} serve all'attore per inserire i dati del proxy;
    \item \textbf{pre-condizioni:} l'utente ha selezionato l'opzione di avviare l'AVD con il proxy;
    \item \textbf{post-condizioni:} i dati sono stati inseriti correttamente;
    \item \textbf{flusso degli eventi principali:}
    \begin{itemize}
        \item l'utente inserisce l'indirizzo IP del server proxy;
        \item l'utente inserisce il numero della porta del server proxy.
    \end{itemize}
\end{itemize}
\subsubsection*{UC-17.4 conferma dei dati}
\begin{itemize}
    \item \textbf{attori:} attore generico;
    \item \textbf{descrizione:} serve all'utente per confermare i dati inseriti per avviare l'AVD;
    \item \textbf{pre-condizioni:} l'utente ha inserito le opzioni di proxy correttamente;
    \item \textbf{post-condizioni:} l'AVD è stato avviato correttamente con le opzioni di proxy;
    \item \textbf{flusso degli eventi principali:}
    \begin{itemize}
        \item l'utente conferma le informazioni inserite e avvia l'AVD.
    \end{itemize}
\end{itemize}

\begin{figure}[H]
    \centering
    \includegraphics[width=10cm, height=8cm]{./immagini/usecase/uc_17.png}
    \caption{Sottocasi d'uso del caso d'uso 17}
\end{figure}
\subsection*{UC-18 Inizio registrazione traffico di rete}\label{subsec:uc-18-inizio-registrazione-traffico-di-rete}
\begin{itemize}
    \item \textbf{attori:} attore generico;
    \item \textbf{descrizione:} quando l'utente vuole registrare le attività di rete effettuate dall'AVD eseguendo l'applicazione, può utilizzare questa funzionalità;
    \item \textbf{pre-condizioni:} l'attore ha avviato l'applicazione;
    \item \textbf{post-condizioni:} l'attore ha registrato le attività di rete e ottenuto un file pcap;
    \item \textbf{flusso degli eventi principali:}
    \begin{itemize}
        \item l'attore seleziona la voce "Start record!";
        \item l'attore successivamente seleziona la voce "Stop Record".
    \end{itemize}
\end{itemize}
\begin{figure}[H]
    \centering
    \includegraphics[width=10cm, height=8cm]{./immagini/usecase/UC-13_14.png}
    \caption{Casi d'uso da 13 a 18}
\end{figure}

%\subsection*{UC-}
%\begin{itemize}
%    \item \textbf{attori:} attore generico;
%    \item \textbf{descrizione:}
%    \item \textbf{pre-condizioni:}
%    \item \textbf{post-condizioni:}
%    \item \textbf{flusso degli eventi principali:}
%    \begin{itemize}
%    \end{itemize}
%\end{itemize}
\input{./capitoli/3_2_requisiti.tex}
\newpage
\section{Tracciamento fonte - requisiti}\label{sec:tracciamento-fonte---requisiti}
Nella seguente tabella sono elencati i casi d'uso con i relativi requisiti derivanti da quest'ultimo.
\begin{center}
    \begin{longtable}{| C{6cm} |C{6cm}|}
        \hline
        \textbf{Fonte} & \textbf{Requisito} \\\hline
        UC-1 &
        \begin{itemize}\itemsep0em
            \item R-1-F-O
            \item R-1.1-F-O
        \end{itemize}
        \\\hline
        UC-2 &
        \begin{itemize}\itemsep0em
            \item R-2-F-O
            \item R-2.1-F-O
            \item R-2.2-F-O
        \end{itemize}
        \\\hline
        UC-3 &
        \begin{itemize}\itemsep0em
            \item R-3-F-O
        \end{itemize} \\\hline
        UC-4 &
        \begin{itemize}\itemsep0em
            \item R-4-F-O
            \item R-4.1-F-D
            \item R-4.2-F-D
            \item R-4.3-F-O
            \item R-4.4-F-O
            \item R-4.5-F-O
            \item R-4.6-F-O
        \end{itemize} \\\hline
        UC-5 &
        \begin{itemize}\itemsep0em
            \item R-6-F-O
        \end{itemize} \\\hline
        UC-6 &
        \begin{itemize}\itemsep0em
            \item R-7-F-O
        \end{itemize} \\\hline
        UC-7 &
        \begin{itemize}\itemsep0em
            \item R-8-F-O
            \item R-8.1-F-O
        \end{itemize} \\\hline
        UC-8 &
        \begin{itemize}\itemsep0em
            \item R-9-F-D
            \item R-9.1-F-O
            \item R-9.2-F-O
            \item R-9.3-F-O
        \end{itemize} \\\hline
        UC-8.2 &
        \begin{itemize}\itemsep0em
            \item R-9.4-F-O
        \end{itemize} \\\hline
        UC-9 & \begin{itemize}\itemsep0em
                   \item R-10-F-O
        \end{itemize} \\\hline
        UC-10 &
        \begin{itemize}\itemsep0em
            \item R-11-F-F
            \item R-11.1-F-F
            \item R-11.2-F-F
            \item R-11.3-F-F
            \item R-11.4-F-F
            \item R-11.4-F-F
        \end{itemize}
        \\\hline

        UC-11 &
        \begin{itemize}\itemsep0em
            \item R-12-F-F
        \end{itemize}
        \\\hline

        UC-12 &
        \begin{itemize}\itemsep0em
            \item R-13-F-O
        \end{itemize}
        \\\hline

        UC-13 &
        \begin{itemize}\itemsep0em
            \item R-14-F-O
        \end{itemize} \\\hline
        UC-14 &
        \begin{itemize}\itemsep0em
            \item R-15-F-O
            \item R-15.1-F-O
            \item R-15.2-F-O
        \end{itemize} \\\hline

        UC-15 &
        \begin{itemize}\itemsep0em
            \item R-16-F-D
            \item R-16.1-F-D
            \item R-16.2-F-D
            \item R-16.3-F-D
        \end{itemize} \\\hline
        UC-17 &
        \begin{itemize}\itemsep0em
            \item R-17-F-D
            \item R-17.1-F-D
            \item R-17.2-F-D
        \end{itemize} \\\hline
        UC-18 &
        \begin{itemize}\itemsep0em
            \item R-18-F-D
            \item R-18.1-F-D
            \item R-18.2-F-D
            \item R-18.3-F-D
        \end{itemize} \\\hline
        Piano di stage &
        \begin{itemize}\itemsep0em
            \item R-1-V-D,
            \item R-2-V-D,
            \item R-3-V-D,
            \item R-4-V-F,
            \item R-1-Q-O,
            \item R-2-Q-O,
            \item R-3-Q-D,
            \item R-4-Q-O
        \end{itemize} \\\hline
        \caption{Tracciamento fonte - requisiti.}
    \end{longtable}
\end{center}

\subsection{Tracciamento requisito - fonte}\label{subsec:tracciamento-requisiti---fonte}
Nella seguente tabella sono presenti i requisiti individuati nella sezione \S3.2, e per ognuno di questi viene indicato il caso d'uso dal quale deriva.
\begin{longtable}{| C{6cm} |C{6cm}|}
    \hline
    \textbf{Requisito} & \textbf{Fonte}  \\\hline
    R-1-F-O            & UC-1            \\\hline
    R-1.1-F-O          & UC-1            \\\hline
    R-2-F-O            & UC-2            \\\hline
    R-2.1-F-O          & UC-2            \\\hline
    R-2.2-F-O          & UC-2            \\\hline
    R-3-F-O            & UC-3            \\\hline
    R-4-F-O            & UC-4            \\\hline
    R-4.1-F-D          & UC-4            \\\hline
    R-5-F-O            & UC-4.1          \\\hline
    R-5.1-F-O          & UC-4.1          \\\hline
    R-5.2-F-O          & UC-4.1          \\\hline
    R-5.3-F-O          & UC-4.1          \\\hline
    R-6-F-O            & UC-5            \\\hline
    R-7-F-O            & UC-6            \\\hline
    R-8-F-O            & UC-7            \\\hline
    R-8.1-F-O          & UC-7            \\\hline
    R-9-F-D            & UC-8            \\\hline
    R-9.1-F-O          & UC-8            \\\hline
    R-9.2-F-O          & UC-8.1          \\\hline
    R-9.3-F-O          & UC-8.1          \\\hline
    R-9.4-F-O          & UC-8.2          \\\hline
    R-10-F-O           & UC-9            \\\hline
    R-11-F-F           & UC-10           \\\hline
    R-11.1-F-F         & UC-10.1         \\\hline
    R-11.2-F-F         & UC-10.2         \\\hline
    R-11.3-F-F         & UC-10.3         \\\hline
    R-11.4-F-F         & UC-10.4         \\\hline
    R-12-F-F           & UC-11           \\\hline
    R-13-F-O           & UC-12           \\\hline
    R-14-F-O           & UC-13           \\\hline
    R-15-F-O           & UC-14           \\\hline
    R-15.1-F-O         & UC-14           \\\hline
    R-15.2-F-O         & UC-14           \\\hline
    R-16-F-D           & UC-15           \\\hline
    R-16.1-F-D         & UC-15           \\\hline
    R-16.2-F-D         & UC-15           \\\hline
    R-16.3-F-D         & UC-15           \\\hline
    R-17-F-D           & UC-17           \\\hline
    R-17.1-F-D         & UC-17           \\\hline
    R-17.2-F-D         & UC-17           \\\hline
    R-18-F-D           & UC-18           \\\hline
    R-18.1-F-D         & UC-18           \\\hline
    R-18.2-F-D         & UC-18           \\\hline
    R-18.3-F-D         & UC-18           \\\hline
    R-19-V-D            & Piano di lavoro \\\hline
    R-20-V-D            & Piano di lavoro \\\hline
    R-21-V-D            & Piano di lavoro \\\hline
    R-22-V-F            & Piano di lavoro \\\hline
    R-23-Q-O            & Piano di lavoro \\\hline
    R-24-Q-O            & Piano di lavoro \\\hline
%        R-2-Q-O& Piano di lavoro\\\hline
%        R-3-Q-D& Piano di lavoro\\\hline
    \caption{Tracciamento requisiti - fonte.}
\end{longtable}




             % Concept Preview
    \input{./capitoli/4_0_progettazione_codifica.tex}             % Product Prototype
    % !TEX encoding = UTF-8
% !TEX TS-program = pdflatex
% !TEX root = ../tesi.tex

%**************************************************************
\chapter{Verifica e validazione}
\label{ch:verifica-validazione}

\section{Analisi statica}\label{sec:analisi-statica}


\section{Test unitari}\label{sec:test-unitari}
In ingegneria del software, per test d’unità si intende l’attività di testing delle
singole unità del software.
Per unità si intende normalmente il componente più piccolo con funzionamento autonomo.
Dipendentemente dal tipo di linguaggio di programmazione, l’unità può essere una funzione, una classe o un metodo.
Come le altre forme di test, i test d’unità possono essere completamente
manuali o automatici, Specialmente nel caso dello unit testing automatico, lo sviluppo dei test case può essere considerato parte integrante dell’attività di sviluppo.\\
Nel caso del progetto \gls{apat} i test unitari sono stati fatti utilizzando il framework JUnit4\footcite{site:junit4} integrato con il tool di \gls{buildautomation}.

\setcounter{rowcount}{0}

\newcounter{testCounter}
\setcounter{testCounter}{0}

\subsection{Specifica dei test}\label{subsec:specifica-dei-test-unitari}
Di seguito sono riportati i test d'unità che verificano il corretto funzionamento delle singole unità.
\begin{longtable}{ |C{3cm} | C{8.5cm}| C{1.5cm} |}
    \hline
    \textbf{Identificativo} &
    \textbf{Descrizione} &
    \textbf{Stato} \\\hline
    \idTest{TU} & Verificare che il file AndroidManifest.xml venga modificato correttamente.
    & I \\\hline
    \idTest{TU} & Verificare che dal file AndroidManifest.xml venga estratto il package corretto.
    & I \\\hline
    \idTest{TU} & Verificare che il nuovo path dell'APK venga aggiornato correttamente.
    & I \\\hline
    \idTest{TU} & Verificare che la lista delle AVD venga ottenuta correttamente.
    & I \\\hline
    \idTest{TU} & Verificare che l'AVD venga avviato con i parametri corretti.
    & I \\\hline
    \idTest{TU} & Verificare che i file dex vengano decompilati correttamente.
    & I \\\hline
    \idTest{TU} & Verificare che elenco dei file presenti nella cartella ./tmp sia corretto.
    & I \\\hline
    \idTest{TU} & Verificare che i file presenti nella cartella ./tmp vengano rimossi correttamente.
    & I \\\hline
    \idTest{TU} & Verificare che la ricompilazione dell'APK avvenga correttamente.
    & I \\\hline
    \idTest{TU} & Verificare che la signing dell'APK ricompilato avvenga correttamente.
    & I \\\hline
    \idTest{TU} & Verificare che l'apk venga installato correttamente nell'AVD.
    & I \\\hline
    \idTest{TU} & Verificare che i dati scaricati dall'AVD siano quelli presenti nell'areas di storage dell'applicazione.
    & I \\\hline
    \idTest{TU} & Verificare che i file DEX vengano decompilati correttamente.
    & I \\\hline
    \idTest{TU} & Verificare che il pdf generato sia corretto.
    & I \\\hline
    \idTest{TU} & Verificare che lo stato del tool caricato sia corretto.
    & I \\\hline
    \idTest{TU} & Verificare che lo stato del tool venga salvato correttamente.
    & I \\\hline
    \idTest{TU} & Verificare che lo stato del tool venga resettato correttamente.
    & I \\\hline
    \idTest{TU} & Verificare che i paragrafi venga inseriti nel pdf correttamente.
    & I \\\hline
    \idTest{TU} & Verificare che il file zip venga estratto correttamente.
    & I \\\hline
    \idTest{TU} & Verificare che il paragrafo generato corretto rispetto all'elenco dei file dati.
    & I \\\hline
    \idTest{TU} & Verificare che il paragrafo generato corretto rispetto all'elenco dei file dati.
    & I \\\hline
    \idTest{TU} & Verificare che il paragrafo generato corretto rispetto all'elenco dei file dati.
    & I \\\hline
    \idTest{TU} & Verificare che il paragrafo generato corretto rispetto all'elenco dei file dati.
    & I \\\hline
    \idTest{TU} & Verificare che il commando generato per decompilare l'APK sia corretto.
    & I \\\hline
    \idTest{TU} & Verificare che il commando generato per ricompilare l'APK sia corretto.
    & I \\\hline
    \idTest{TU} & Verificare che il commando generato per firmare l'APK ricompilato sia corretto.
    & I \\\hline
    \idTest{TU} & Verificare che il commando generato per ottenere l'elenco delle AVD sia corretto.
    & I \\\hline
    \idTest{TU} & Verificare che il commando generato per avviare l'AVD sia corretto.
    & I \\\hline
    \idTest{TU} & Verificare che il commando generato per installare l'APK sull'AVD sia corretto.
    & I \\\hline
    \idTest{TU} & Verificare che il commando generato per scaricare l'area di storage dell'app installato sia corretto.
    & I \\\hline
    \idTest{TU} & Verificare che il commando generato per convertire i file .class in .java sia corretto.
    & I \\\hline
    \idTest{TU} & Verificare che il commando generato per rimuovere i file decompilati sia corretto.
    & I \\\hline
    \idTest{TU} & Verificare che il commando generato per connettersi all'AVD come root sia corretto.
    & I \\\hline
    \idTest{TU} & Verificare che il commando generato per connettersi all'AVD normalmente sia corretto.
    & I \\\hline
    \idTest{TU} & Verificare che il commando generato per ottenere i parametri del terminale sia corretto.
    & I \\\hline
    \idTest{TU} & Verificare che la mappa restituita contenga esattamente le scelte dell'utente.
    & I \\\hline
    \idTest{TU} & Verificare che vengano accettati solo i file di tipo APK.
    & I \\\hline
    \idTest{TU} & Verificare che la descrizione restituita sia corretta.
    & I \\\hline
    \idTest{TU} & Verificare che vengano accettati solo i file di tipo KJS.
    & I \\\hline
    \idTest{TU} & Verificare che la descrizione restituita sia corretta.
    & I \\\hline
    \idTest{TU} & Verificare che vengano accettati solo i file di tipo TXT.
    & I \\\hline
    \idTest{TU} & Verificare che la descrizione restituita sia corretta.
    & I \\\hline
    \idTest{TU} & Verificare che la stringa restituita sia corretta.
    & I \\\hline
    \idTest{TU} & Verificare che l'elenco dei file ottenuto sia corretto.
    & I \\\hline
    \idTest{TU} & Verificare che l'arrotondamento avviene correttamente.
    & I \\\hline
    \caption{Test d'unità}
\end{longtable}
\setcounter{rowcount}{0}

\subsection{Tracciamento}\label{subsec:tracciamento-unitari}
La seguente tabella associa ogni singolo test unitario all'unità verificata.
\begin{longtable}{ |C{3cm} |C{10.5cm}|}
    \hline
    \textbf{Identificativo} &
    \textbf{Componente} \\\hline
    \idTest{TU} & AndroidManifest.editDebugAttribute()       \\\hline
    \idTest{TU} & AndroidManifest.getPackageName()           \\\hline
    \idTest{TU} & Controller.updateApkPath()                 \\\hline
    \idTest{TU} & Controller.refreshAVDList()                \\\hline
    \idTest{TU} & Controller.startAvd()                      \\\hline
    \idTest{TU} & Controller.decompile()                     \\\hline
    \idTest{TU} & Controller.listFolderElements()            \\\hline
    \idTest{TU} & Controller.removeDecompiledFiles()         \\\hline
    \idTest{TU} & Controller.recompile()                     \\\hline
    \idTest{TU} & Controller.signAPK()                       \\\hline
    \idTest{TU} & Controller.installCompiledApk()            \\\hline
    \idTest{TU} & Controller.dumpDataFromAVD()               \\\hline
    \idTest{TU} & Controller.decodeDex()                     \\\hline
    \idTest{TU} & Controller.analyze()                       \\\hline
    \idTest{TU} & Controller.loadState()                     \\\hline
    \idTest{TU} & Controller.saveState()                     \\\hline
    \idTest{TU} & Controller.resetModelState()               \\\hline
    \idTest{TU} & PDFWriter.addParagraphs()                  \\\hline
    \idTest{TU} & Unzipper.unzip()                           \\\hline
    \idTest{TU} & DumpDataBase.doAnalysis()                  \\\hline
    \idTest{TU} & DumpedFilesAnalyzer.doAnalysis()           \\\hline
    \idTest{TU} & LambdaCounter.doAnalysis()                 \\\hline
    \idTest{TU} & StringFinder.doAnalysis()                  \\\hline
    \idTest{TU} & Commands.getDecompileCommand()             \\\hline
    \idTest{TU} & Commands.getCompileApkCommand()            \\\hline
    \idTest{TU} & Commands.getSigningApkCommand()            \\\hline
    \idTest{TU} & Commands.getListOfAvds()                   \\\hline
    \idTest{TU} & Commands.getStartAvdCommand()              \\\hline
    \idTest{TU} & Commands.getInstallAPK()                   \\\hline
    \idTest{TU} & Commands.getDumpApplicationDataCommand()   \\\hline
    \idTest{TU} & Commands.getConvertToJavaCommand()         \\\hline
    \idTest{TU} & Commands.getRemoveDecompiledFilesCommand() \\\hline
    \idTest{TU} & Commands.connectToAdbAsRoot()              \\\hline
    \idTest{TU} & Commands.getAttachToAvdCommand()           \\\hline
    \idTest{TU} & Commands.getCli()                          \\\hline
    \idTest{TU} & Analysis.chooser()                         \\\hline
    \idTest{TU} & APKFilter.accept()                         \\\hline
    \idTest{TU} & APKFilter.getDescription()                 \\\hline
    \idTest{TU} & KeystoreFilter.accept()                    \\\hline
    \idTest{TU} & KeystoreFilter.getDescription()            \\\hline
    \idTest{TU} & TxtFilter.accept()                         \\\hline
    \idTest{TU} & TxtFilter.getDescription()                 \\\hline
    \idTest{TU} & Utils.getDate()                            \\\hline
    \idTest{TU} & Utils.listAllFiles()                       \\\hline
    \idTest{TU} & Utils.round()                              \\\hline
    \caption{Tracciamento dei test d'unità}
\end{longtable}

I test sopraelencati sono stati superati con successo, con una \gls{codecoverage} del \textit{84.2\%}.
\newpage

             % Product Design Freeze e SOP
    % !TEX encoding = UTF-8
% !TEX TS-program = pdflatex
% !TEX root = ../tesi.tex

%**************************************************************
\chapter{Conclusioni}
\label{ch:conclusioni}
%**************************************************************
\intro{}

% !TEX encoding = UTF-8
% !TEX TS-program = pdflatex
% !TEX root = ../tesi.tex

%**************************************************************


\section{Consuntivo finale}\label{sec:consuntivo-finale}
Il lavoro è stato svolto nei tempi previsti.
Il numero di ore pianificato per ognuna delle fasi di sviluppo è stato sufficiente.
Nella seguente tabella sono riportate le attività svolte con le ore previste e, tra parentesi, quelle effettive.

\begin{longtable}{|C{3cm}|C{8cm}|C{2cm}|}
    \hline
    \textbf{Attività} &
    \textbf{Descrizione} &
    \textbf{Previste} (\textbf{Effettive}) \\\hline
    Formazione     & Formazione sulle tecnologie.
    & 40 (35) \\\hline
    Pianificazione & Pianificazione delle attività da svolgere.
    & 40 (38)    \\\hline
    Analisi dei requisiti & Individuazione dei casi d'uso, dei requisiti e creazione delle tabelle per il tracciamento dei requisiti.
    & 40 (37) \\\hline

    Progettazione & Progettazione del tool APAT includendo:
    \begin{itemize}%\itemsep0em
        \item progettazione della logica per creazione dei comandi;
        \item progettazione della logica dell'esecuzione dei comandi generati;
        \item progettazione della logica per effettuare l'analisi del codice sorgente;
        \item progettazione dell'interfaccia grafica.
    \end{itemize}
    &60 (50)\\\hline
    Codifica & Codifica del tool APAT includendo:
    \begin{itemize}%\itemsep0em
        \item codifica della logica per creazione dei comandi;
        \item codifica della logica dell'esecuzione dei comandi generati;
        \item codifica della logica per effettuare l'analisi del codice sorgente;
        \item codifica dell'interfaccia grafica.
    \end{itemize}
    & 100 (118) \\\hline
    Test           & Analisi statica, stesura dei test automatici, test manuali del tool.
    & 20 (22)   \\\hline
    Collaudo       & User Acceptance Test, verifica soddisfacimento requisiti.
    & 20 (20)    \\\hline
    \caption{Attività svolte}
\end{longtable}

La pianificazione è stata svolta in maniera abbastanza precisa.
In questo modo, le attività sono state svolte nei tempi previsti della durata dello stage.
Nonostante si fossero presentate alcune difficoltà la pianificazione non ha subito gravi ritardi grazie all'aiuto del tutor aziendale.


% !TEX encoding = UTF-8
% !TEX TS-program = pdflatex
% !TEX root = ../tesi.tex

%**************************************************************
\section{Raggiungimento degli obiettivi}\label{sec:raggiungimento-degli-obiettivi}

% !TEX encoding = UTF-8
% !TEX TS-program = pdflatex
% !TEX root = ../tesi.tex

%**************************************************************
\section{Conoscenze acquisite}\label{sec:conoscenze-acquisite}
Durante lo svolgimento dello stage sono state acquisite delle nuove conoscenze, come:
\begin{itemize}\itemsep0em
    \item la tecnica del reverse engineering applicata alle applicazioni Android;
    \item l'analisi del codice sorgente, dei dati presenti nell'area di storage e nei database SQLite possono rivelare dati sensibili e dati personali;
    \item applicazione di alcuni dei design pattern visti a lezione;
    \item libreria Swing di Java;
    \item alcuni aspetti di Maven, tool di build automation;
    \item utilizzo dell'AVD per monitorare le attività di rete dell'applicazione.
\end{itemize}
%Inoltre, lavorando insieme al tutor aziendale la differenza tra il mondo lavorativo e quello accademico si è evidenziata in modo evidente, e

% !TEX encoding = UTF-8
% !TEX TS-program = pdflatex
% !TEX root = ../tesi.tex

%**************************************************************
\section{Valutazione personale}\label{sec:valutazione-personale}
Ritengo che concludere un percorso formativo come quello d'Informatica con uno stage in un'azienda è molto interessante e utile, poiché è un'occasione importante per mettere in pratica le conoscenze teoriche apprese durante i corsi d'insegnanmento, adattare il proprio metodo di lavoro alla realtà aziendale.\\
Inoltre, quest'esperienza mi ha permesso di entrare in contatto con gli esperti del settore informatico, che mi hanno fatto conoscere molti aspetti lavorativi che prima di allora ignoravo completamente.
Infine, penso che questo stage è stato costruttivo sia nella formazione professionale che personale e ho apprezzato molto l'ospitalità e la disponibilità dell'azienda Imola Informatica S.p.A. e del tutor aziendale Alessandro Proscia per avermi fornito gli aiuti necessari per portare al termine il progetto.
             % Conclusioni
%**************************************************************
% Materiale finale
%**************************************************************
    \backmatter
    \printglossaries
    \input{inizio-fine/bibliografia}
\end{document}
