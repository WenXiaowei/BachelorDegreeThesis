\section{Test unitari}\label{sec:test-unitari}
In ingegneria del software, per test d'unità si intende l'attività di testing delle singole parte unità del software.
Per unità si intende normalmente il componente più piccolo con funzionamento autonoma, quindi dipendentemente dal tipo di linguaggio di programmazione, l'unità può essere una funzione, una classe o un metodo.
Come le altre forme di test, i test d'unità può variare da completamente manuale ad automatico.
Specialmente nel caso dello unit testing automatico, lo sviluppo dei test case può essere considerato parte integrante dell'attività di sviluppo.\\

Nel caso del progetto \gls{apat} i test unitari sono stati fatti utilizzando il framework JUnit4\footcite{site:junit4} integrato con il tool di \gls{buildautomation} Maven\footcite{site:maven}.
\setcounter{rowcount}{0}

\newcounter{testCounter}
\setcounter{testCounter}{0}

\subsection{Specifica dei test}\label{subsec:specifica-dei-test-unitari}
Di seguito sono riportati i test d'unità che verificano il corretto funzionamento delle singole unità.
\begin{center}
    \begin{longtable}{ |C{3cm} | C{8.5cm}| C{1.5cm} |}
        \hline
        \textbf{Identificativo} &
        \textbf{Descrizione} &
        \textbf{Stato} \\\hline
        \idTest{TU} & Verificare che il file AndroidManifest.xml venga modificato correttamente.
        & I \\\hline
        \idTest{TU} & Verificare che dal file AndroidManifest.xml venga estratto il package corretto.
        & I \\\hline
        \idTest{TU} & Verificare che il nuovo path dell'APK venga aggiornato correttamente.
        & I \\\hline
        \idTest{TU} & Verificare che la lista delle AVD venga ottenuta correttamente.
        & I \\\hline
        \idTest{TU} & Verificare che l'AVD venga avviato con i parametri corretti.
        & I \\\hline
        \idTest{TU} & Verificare che i file dex vengano decompilati correttamente.
        & I \\\hline
        \idTest{TU} & Verificare che elenco dei file presenti nella cartella ./tmp sia corretto.
        & I \\\hline
        \idTest{TU} & Verificare che i file presenti nella cartella ./tmp vengano rimossi correttamente.
        & I \\\hline
        \idTest{TU} & Verificare che la ricompilazione dell'APK avvenga correttamente.
        & I \\\hline
        \idTest{TU} & Verificare che la signing dell'APK ricompilato avvenga correttamente.
        & I \\\hline
        \idTest{TU} & Verificare che l'apk venga installato correttamente nell'AVD.
        & I \\\hline
        \idTest{TU} & Verificare che i dati scaricati dall'AVD siano quelli presenti nell'areas di storage dell'applicazione.
        & I \\\hline
        \idTest{TU} & Verificare che i file DEX vengano decompilati correttamente.
        & I \\\hline
        \idTest{TU} & Verificare che il pdf generato sia corretto.
        & I \\\hline
        \idTest{TU} & Verificare che lo stato del tool caricato sia corretto.
        & I \\\hline
        \idTest{TU} & Verificare che lo stato del tool venga salvato correttamente.
        & I \\\hline
        \idTest{TU} & Verificare che lo stato del tool venga resettato correttamente.
        & I \\\hline
        \idTest{TU} & Verificare che i paragrafi venga inseriti nel pdf correttamente.
        & I \\\hline
        \idTest{TU} & Verificare che il file zip venga estratto correttamente.
        & I \\\hline
        \idTest{TU} & Verificare che il paragrafo generato corretto rispetto all'elenco dei file dati.
        & I \\\hline
        \idTest{TU} & Verificare che il paragrafo generato corretto rispetto all'elenco dei file dati.
        & I \\\hline
        \idTest{TU} & Verificare che il paragrafo generato corretto rispetto all'elenco dei file dati.
        & I \\\hline
        \idTest{TU} & Verificare che il paragrafo generato corretto rispetto all'elenco dei file dati.
        & I \\\hline
        \idTest{TU} & Verificare che il commando generato per decompilare l'APK sia corretto.
        & I \\\hline
        \idTest{TU} & Verificare che il commando generato per ricompilare l'APK sia corretto.
        & I \\\hline
        \idTest{TU} & Verificare che il commando generato per firmare l'APK ricompilato sia corretto.
        & I \\\hline
        \idTest{TU} & Verificare che il commando generato per ottenere l'elenco delle AVD sia corretto.
        & I \\\hline
        \idTest{TU} & Verificare che il commando generato per avviare l'AVD sia corretto.
        & I \\\hline
        \idTest{TU} & Verificare che il commando generato per installare l'APK sull'AVD sia corretto.
        & I \\\hline
        \idTest{TU} & Verificare che il commando generato per scaricare l'area di storage dell'app installato sia corretto.
        & I \\\hline
        \idTest{TU} & Verificare che il commando generato per convertire i file .class in .java sia corretto.
        & I \\\hline
        \idTest{TU} & Verificare che il commando generato per rimuovere i file decompilati sia corretto.
        & I \\\hline
        \idTest{TU} & Verificare che il commando generato per connettersi all'AVD come root sia corretto.
        & I \\\hline
        \idTest{TU} & Verificare che il commando generato per connettersi all'AVD normalmente sia corretto.
        & I \\\hline
        \idTest{TU} & Verificare che il commando generato per ottenere i parametri del terminale sia corretto.
        & I \\\hline
        \idTest{TU} & Verificare che la mappa restituita contenga esattamente le scelte dell'utente.
        & I \\\hline
        \idTest{TU} & Verificare che vengano accettati solo i file di tipo APK.
        & I \\\hline
        \idTest{TU} & Verificare che la descrizione restituita sia corretta.
        & I \\\hline
        \idTest{TU} & Verificare che vengano accettati solo i file di tipo KJS.
        & I \\\hline
        \idTest{TU} & Verificare che la descrizione restituita sia corretta.
        & I \\\hline
        \idTest{TU} & Verificare che vengano accettati solo i file di tipo TXT.
        & I \\\hline
        \idTest{TU} & Verificare che la descrizione restituita sia corretta.
        & I \\\hline
        \idTest{TU} & Verificare che la stringa restituita sia corretta.
        & I \\\hline
        \idTest{TU} & Verificare che l'elenco dei file ottenuto sia corretto.
        & I \\\hline
        \idTest{TU} & Verificare che l'arrotondamento avviene correttamente.
        & I \\\hline
        \caption{Test d'unità}
    \end{longtable}
\end{center}
\setcounter{rowcount}{0}

\subsection{Tracciamento}\label{subsec:tracciamento-unitari}
La seguente tabella associa ogni singolo test unitario all'unità verificata.
\begin{center}
    \begin{longtable}{ |C{3cm} |C{10.5cm}|}
        \hline
        \textbf{Identificativo} &
        \textbf{Componente} \\\hline
        \idTest{TU} & AndroidManifest.editDebugAttribute()       \\\hline
        \idTest{TU} & AndroidManifest.getPackageName()           \\\hline
        \idTest{TU} & Controller.updateApkPath()                 \\\hline
        \idTest{TU} & Controller.refreshAVDList()                \\\hline
        \idTest{TU} & Controller.startAvd()                      \\\hline
        \idTest{TU} & Controller.decompile()                     \\\hline
        \idTest{TU} & Controller.listFolderElements()            \\\hline
        \idTest{TU} & Controller.removeDecompiledFiles()         \\\hline
        \idTest{TU} & Controller.recompile()                     \\\hline
        \idTest{TU} & Controller.signAPK()                       \\\hline
        \idTest{TU} & Controller.installCompiledApk()            \\\hline
        \idTest{TU} & Controller.dumpDataFromAVD()               \\\hline
        \idTest{TU} & Controller.decodeDex()                     \\\hline
        \idTest{TU} & Controller.analyze()                       \\\hline
        \idTest{TU} & Controller.loadState()                     \\\hline
        \idTest{TU} & Controller.saveState()                     \\\hline
        \idTest{TU} & Controller.resetModelState()               \\\hline
        \idTest{TU} & PDFWriter.addParagraphs()                  \\\hline
        \idTest{TU} & Unzipper.unzip()                           \\\hline
        \idTest{TU} & DumpDataBase.doAnalysis()                  \\\hline
        \idTest{TU} & DumpedFilesAnalyzer.doAnalysis()           \\\hline
        \idTest{TU} & LambdaCounter.doAnalysis()                 \\\hline
        \idTest{TU} & StringFinder.doAnalysis()                  \\\hline
        \idTest{TU} & Commands.getDecompileCommand()             \\\hline
        \idTest{TU} & Commands.getCompileApkCommand()            \\\hline
        \idTest{TU} & Commands.getSigningApkCommand()            \\\hline
        \idTest{TU} & Commands.getListOfAvds()                   \\\hline
        \idTest{TU} & Commands.getStartAvdCommand()              \\\hline
        \idTest{TU} & Commands.getInstallAPK()                   \\\hline
        \idTest{TU} & Commands.getDumpApplicationDataCommand()   \\\hline
        \idTest{TU} & Commands.getConvertToJavaCommand()         \\\hline
        \idTest{TU} & Commands.getRemoveDecompiledFilesCommand() \\\hline
        \idTest{TU} & Commands.connectToAdbAsRoot()              \\\hline
        \idTest{TU} & Commands.getAttachToAvdCommand()           \\\hline
        \idTest{TU} & Commands.getCli()                          \\\hline
        \idTest{TU} & Analysis.chooser()                         \\\hline
        \idTest{TU} & APKFilter.accept()                         \\\hline
        \idTest{TU} & APKFilter.getDescription()                 \\\hline
        \idTest{TU} & KeystoreFilter.accept()                    \\\hline
        \idTest{TU} & KeystoreFilter.getDescription()            \\\hline
        \idTest{TU} & TxtFilter.accept()                         \\\hline
        \idTest{TU} & TxtFilter.getDescription()                 \\\hline
        \idTest{TU} & Utils.getDate()                            \\\hline
        \idTest{TU} & Utils.listAllFiles()                       \\\hline
        \idTest{TU} & Utils.round()                              \\\hline
        \caption{Tracciamento dei test d'unità}
    \end{longtable}
\end{center}

I test sopraelencati sono stati superati con successo, con una \gls{codecoverage} del \textit{84.2\%}.
\newpage