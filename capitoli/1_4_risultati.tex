% !TEX encoding = UTF-8
% !TEX TS-program = pdflatex
% !TEX root = ../tesi.tex
\section{Risultati}\label{sec:risultati}
Lo stage si è svolto dal 4 maggio 2020 al 26 giugno 2020, la durata complessiva è di 312 ore lavorative.\\
Gli obiettivi fissati nella sezione \S1.2 sono stati tutti raggiunti, il tutor aziendale Alessandro Proscia si è mostrato soddisfatto del prodotto finale.
Durante lo stage ho sviluppato un software per l'analisi dei file APK che è in grado di decompilare l'APK, analizzare le risorse contenute al suo interno, ricompilarlo, installarlo su un AVD e quindi monitorare il traffico di rete generato dall'applicazione tutto questo in modo automatico.
Inoltre, genera un PDF che contiene i risultati ottenuti durante l'analisi.\\
Il processo di analisi, testato su quattro applicazione, viene eseguito mediamente in 131.69 secondi, ma esiste una consistente differenza in base alla dimensione dell'applicazione e al linguaggio utilizzato per la sua realizzazione.
Questo divario è causato dagli strumenti di terze di parti impiegati, per cui non è stato possibile ridurla.
\\
Per la sua realizzazione sono state utilizzate le seguenti tecnologie:
\begin{itemize}
    \item CFR: decompilatore Java;
    \item APKTool: decompilatore APK;
    \item Dex2Jar: decompilatore Dex verso il formato jar;
    \item XPath: linguaggio per estrarre le informazioni dal file AndroidManifest.xml;
    \item Java: linguaggio di programmazione utilizzato per la realizzazione del tool;
    \item Android Emulator: emulatore dei dispositivi Android, utilizzato per eseguire l'applicazione.
\end{itemize}
