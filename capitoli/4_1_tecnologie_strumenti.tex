% !TEX encoding = UTF-8
% !TEX TS-program = pdflatex
% !TEX root = ../tesi.tex


\section{Tecnologie}\label{sec:tecnologie}
Ecco una panoramica delle tecnologie e degli strumenti utilizzati.
\begin{center}
    \begin{longtable}{ | C{3cm}| C{3cm} |C{7cm} |}
        \hline
        \textbf{Strumento Tecnologia} &
        \textbf{Consigliato dall'azienda}&
        \textbf{Motivazioni della scelta} \\\hline
        CFR & Sì & - \\\hline
        ApkTool& Sì & - \\\hline
        Dex2Jar& Sì & - \\\hline
        XPath& No & La scelta di utilizzare XPath è dovuta alle necessità di dover interagire con un documento in formato XML, l'utilizzo di XPath ha semplificato di l'estrazione dei dati dal documento AndroidManifest.xml.\\\hline
        Java& No & È stato scelto questo linguaggio di programmazione perché è bene conosciuto dallo stagista, offre una libreria per le interfacce grafiche, è portabile, ciò ha consentito la creazione dello strumento anche per la piattaforma Linux.\\\hline
        Intellij Idea& No & Essendo un IDE molto potente, permette d'interagire con lo strumento di \gls{buildautomation} e di effettuare la compilazione del codice sorgente.\\\hline
        Android Emulator& Sì & - \\\hline
        Maven& Sì & - \\\hline
        Astah UML& No & Astah UML è un prodotto software per la realizzazione dei diagrammi UML già conosciuto dallo stagista in quanto è utilizzato in precedenza.\\\hline
        GitLab& Sì & -\\\hline
        \caption{Panoramica tecnologie e strumenti utilizzati.}
    \end{longtable}
\end{center}

Di seguito, ogni tecnologia/strumento presente nella Tabella 4.1 verrà presentato con maggiori dettagli.
\subsection*{CFR - decompilatore java}
Un decompilatore è un software che prende come input un file eseguibile e tenta di creare del codice sorgente ad alto livello che può essere ricompilato successivamente.
È lo strumento che effettua l'operazione inversa del compilatore.
I decompilatori solitamente non riescono a ricostruire il codice sorgente nella forma originale, ma riescono spesso a ottenere del codice offuscato.
Con offuscamento del codice si intende la tecnica che viene applicata al codice sorgente o codice macchina per rendere il codice meno leggibile in modo da rendere l'operazione di reverse engineering più complicata.
Nonostante ciò, il decompilatore rimane uno dei più importanti tool utilizzati nell'ambito del reverse engineering del software.
CFR è il decompilatore utilizzato per trasformare il codice bytecode \textit{.class} in codice sorgente \textit{.java};
\url{http://www.benf.org/other/cfr/index.html}

\subsection*{Apk Tool}
Apktool è un tool per effettuare il reverse engineering delle applicazioni Android.
Può decodificare le risorse contenute nell'APK in una forma quasi uguale a quella originale e può ricostruire l'APK dopo le modifiche effettuate alle risorse modificate.
Nel tool APAT è stato utilizzato per poter decompattare un'APK in modo da ottenere i file che compongono l'APK originale.
Apktool permette anche di ricomporre un'apk utilizzando le risorse ottenute precedentemente.
\url{https://ibotpeaches.github.io/Apktool/}
\begin{figure}[H]
    \centering
    \includegraphics[width=6cm, height=3cm]{./immagini/apktool.png}
    \caption{Icona Apktool.}\label{fig:apktool}
\end{figure}

\subsection*{Dex2Jar}
Questo strumento permette di trasformare i file \textit{.dex} in formato \textit{.jar} in modo
che il contenuto del file possa essere letto attraverso un software di compressione, come 7Zip.
\url{https://github.com/pxb1988/dex2jar}

\subsection*{XPath}\label{subsec:xpath}
In informatica XPath è un linguaggio, parte della famiglia \gls{xml}, che permette d'individuare i nodi all'interno di un documento XML. Le espressioni XPath, a differenza delle espressioni XML, non servono a identificare la struttura di un documento, bensì a localizzarne con precisione i nodi.
Nell'APAT XPath è stato utilizzato per poter estrapolare alcuni dati necessari per poter effettuare il dump dei dati dall'area di storage dell'app e ottenere le informazioni per poter decompilare solamente nel package dell'applicazione.

\subsection*{Java}
In informatica, Java\cite{womak:effective-java} è un linguaggio di programmazione ad alto livello, orientato agli oggetti e tipizzazione statica, che si appoggia sull'omonima piattaforma software di esecuzione, specificamente progettato per essere il più possibile indipendente dal sistema hardware di esecuzione.
Nell'APAT è stato utilizzato per la stesura del codice, in particolare, per l'interfaccia grafica è stata utilizzata la libreria di Java, chiamata \gls{swing}.
\begin{figure}[H]
    \centering
    \includegraphics[width=8cm, height=5cm]{./immagini/java.jpg}
    \caption{Icona Java.}\label{fig:java}
\end{figure}

\subsection*{JSON}
JSON (JavaScript Object Notation) è un semplice formato per lo scambio di dati.
Per le persone è facile da leggere e scrivere, mentre per le macchine risulta facile generare il codice e analizzarne la sintassi.

JSON è basato su due strutture:
\begin{itemize}
    \item un insieme di coppie nome/valore, in diversi linguaggi, questo è realizzato come un oggetto, un record, uno struct, un dizionario, una tabella hash, un elenco di chiavi o un array associativo.
    \item un elenco ordinato di valori, nella maggior parte dei linguaggi questo si realizza con un array, un vettore o una sequenza.
\end{itemize}
Nell'APAT è stato utilizzato per contenere i dati di stato del tool quando viene chiuso e alcuni dati che indicano i percorsi ai tool di terze parti.
\url{https://www.json.org/json-it.html}
\begin{figure}[H]
    \centering
    \includegraphics[width=5cm, height=5cm]{./immagini/json.png}
    \caption{Icona JSON.}\label{fig:json}
\end{figure}


\section{Strumenti}\label{sec:strumenti}

\subsection*{Intellij IDEA}
Intellij IDEA è l'\gls{ide} che è stato utilizzato per la scrittura del codice in Java sviluppato da JetBrains.
\url{https://www.jetbrains.com/idea/}
\begin{figure}[H]
    \centering
    \includegraphics[width=5cm, height=5cm]{./immagini/intellij.png}
    \caption{Icona Intellij Idea.}\label{fig:intellij}
\end{figure}

\subsection*{Android Emulator}
Android emulator è lo strumento che permette di avviare degli emulatori Android, conosciuti anche come \gls{avd}.
È stato utilizzato per eseguire l'applicazione ricompilata, per poter fare il dump dei dati presenti nell'area di storage dell'applicazione, come i file JSON, XML e SQLite, o come il file delle attività di rete.
\url{https://developer.android.com/studio/run/emulator}
\begin{figure}[H]
    \centering
    \includegraphics[width=4cm, height=4cm]{./immagini/emulator.png}
    \caption{Icona Android Emulator.}\label{fig:emulator}
\end{figure}

\subsection*{Maven}
Maven è un tool di \gls{buildautomation}, basato sul concetto di Project Object Model (POM). Maven può gestire le build, generare i report e la documentazione da un singolo centro d'informazione.
\url{https://maven.apache.org/}
\begin{figure}[H]
    \centering
    \includegraphics[width=6cm, height=3cm]{./immagini/maven.png}
    \caption{Icona Maven.}\label{fig:maven}
\end{figure}
Nel caso dell'APAT è stato utilizzato per la build del tool, generando in questo modo il file jar.

\subsection*{Astah UML}
Astah è uno strumento di modellazione \gls{uml}. Permette di creare vari diagrammi UML, come quelli di classe, di package, di sequenza e di attività.
\url{https://astah.net/products/astah-community/}
\begin{figure}[H]
    \centering
    \includegraphics[width=6cm, height=3cm]{./immagini/astah.png}
    \caption{Icona Astah UML.}\label{fig:astah}
\end{figure}

\subsection*{Git - GitLab}
Gitlab è una piattaforma web open-source che permette la gestione di repository e di funzioni d'\gls{its}.
Questa è un'istanza privata dell'azienda Imola Informatica S.p.A..
\url{https://git.imolinfo.it/users/sign_in}
\begin{figure}[H]
    \centering
    \includegraphics[width=6cm, height=3cm]{./immagini/gitlab.png}
    \caption{Icona GitLab.}\label{fig:gitlab}
\end{figure}
