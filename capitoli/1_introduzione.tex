% !TEX encoding = UTF-8
% !TEX TS-program = pdflatex
% !TEX root = ../tesi.tex

%**************************************************************
\chapter{Introduzione}\label{ch:introduzione}
%**************************************************************

Introduzione al contesto applicativo.\\

\noindent Esempio di utilizzo di un termine nel glossario \\
\gls{api}. \\

\noindent Esempio di citazione in linea \\
\cite{site:agile-manifesto}. \\

\noindent Esempio di citazione nel pie' di pagina \\
citazione\footcite{womak:lean-thinking} \\

%**************************************************************
\section{L'azienda}

Imola Informatica è una società indipendente di consulenza IT.

Entra in gioco ogni volta in cui una azienda pubblica o privata vuole migliorare i propri servizi. Innovare i propri processi di lavoro, gli approcci di management per cogliere le opportunità business offerte dalla trasformazione digitale.

\`{E} a servizio dei principali gruppi finanziari e assicurativi e ogni giorno è a fianco di grandi aziende e piccole startup nel gestire il cambiamento tecnologico e culturale.

Fa parte di una rete che condivide l’idea di fare innovazione a misura delle persone, delle imprese e della collettività. Sono impegnati per lo sviluppo consapevole di comunità locali e smart cities.

%**************************************************************
\section{L'idea}

Introduzione all'idea dello stage.

%**************************************************************
\section{Organizzazione del testo}

\begin{description}
    \item[{\hyperref[cap:processi-metodologie]{Il secondo capitolo}}] descrive ...
    
    \item[{\hyperref[cap:descrizione-stage]{Il terzo capitolo}}] approfondisce ...
    
    \item[{\hyperref[cap:analisi-requisiti]{Il quarto capitolo}}] approfondisce ...
    
    \item[{\hyperref[cap:progettazione-codifica]{Il quinto capitolo}}] approfondisce ...
    
    \item[{\hyperref[cap:verifica-validazione]{Il sesto capitolo}}] approfondisce ...
    
    \item[{\hyperref[cap:conclusioni]{Nel settimo capitolo}}] descrive ...
\end{description}

Riguardo la stesura del testo, relativamente al documento sono state adottate le seguenti convenzioni tipografiche:
\begin{itemize}
	\item gli acronimi, le abbreviazioni e i termini ambigui o di uso non comune menzionati vengono definiti nel glossario, situato alla fine del presente documento;
	\item per la prima occorrenza dei termini riportati nel glossario viene utilizzata la seguente nomenclatura: \emph{parola}\glsfirstoccur;
	\item i termini in lingua straniera o facenti parti del gergo tecnico sono evidenziati con il carattere \emph{corsivo}.
\end{itemize}