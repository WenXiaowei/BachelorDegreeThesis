% !TEX encoding = UTF-8
% !TEX TS-program = pdflatex
% !TEX root = ../tesi.tex

%**************************************************************
\chapter{Introduzione}\label{ch:introduzione}
%**************************************************************

In questa sezione viene fatta una breve presentazione dell'azienda Imola Informatica S.p.A e una breve introduzione alle idee dello stage.
%\noindent Esempio di utilizzo di un termine nel glossario \\
%\gls{api}. \\
%
%\noindent Esempio di citazione in linea \\
%\cite{site:agile-manifesto}. \\
%
%\noindent Esempio di citazione nel pie' di pagina \\
%citazione\footcite{womak:lean-thinking} \\

%**************************************************************
\section{L'azienda}\label{sec:l'azienda}

Imola Informatica è una società indipendente di consulenza IT, entra in gioco ogni volta in cui una azienda pubblica o privata vuole migliorare i propri servizi. Innovare i propri processi di lavoro, gli approcci di management per cogliere le opportunità business offerte dalla trasformazione digitale.

È a servizio dei principali gruppi finanziari e assicurativi e ogni giorno è a fianco di grandi aziende e piccole startup nel gestire il cambiamento tecnologico e culturale.

Fa parte di una rete che condivide l’idea di fare innovazione a misura delle persone, delle imprese e della collettività. Sono impegnati per lo sviluppo consapevole di comunità locali e smart cities.

%**************************************************************
\section{L'idea}\label{sec:l'idea}

Le applicazioni mobili rappresentano oggi una delle principali sfide per la sicurezza informatica: negli ultimi anni se la rapida diffusione degli smartphone ha visto emergere il mobile come uno dei principali canali per l'erogazione di servizi, sono aumentati esponenzialmente gli attacchi contro le piattaforme mobili.

Le applicazioni mobili soffrono spesso di debolezze intrinseche dovute al design dell'applicazione e al suo sviluppo, debolezze che possono prevedere la memorizzazione d'informazioni sensibili sul device, possibilità di modificare l'applicazione (mediante \gls{repackaging} dell'App), o la possibilità di un semplice reverse engineering.

%**************************************************************
\section{Organizzazione del testo}\label{sec:organizzazione-del-testo}

\begin{description}
    \item[{\hyperref[ch:introduzione]{Il primo capitolo}}] dà un'introduzione generale all'azienda ospitante e l'idea fondante del progetto di stage;

    \item[{\hyperref[ch:descrizione-stage]{Il secondo capitolo}}] approfondisce gli obiettivi del progetto con i relativi requisiti e pianificazione;

    \item[{\hyperref[ch:analisi-requisiti]{Il terzo capitolo}}] approfondisce l'analisi dei requisiti del tool;

    \item[{\hyperref[ch:progettazione-e-codifica]{Il quarto capitolo}}] approfondisce la progettazione e la codifica del tool;
    
    \item[{\hyperref[ch:verifica-validazione]{Il quinto capitolo}}] approfondisce la verifica e la validazione del tool.
    
    \item[{\hyperref[ch:conclusioni]{Nel sesto capitolo}}] descrive le opinioni personali del laureando.
\end{description}

Riguardo la stesura del testo, relativamente al documento sono state adottate le seguenti convenzioni tipografiche:
\begin{itemize}
	\item gli acronimi, le abbreviazioni e i termini ambigui o di uso non comune menzionati vengono definiti nel glossario, situato alla fine del presente documento;
	\item per la prima occorrenza dei termini riportati nel glossario viene utilizzata la seguente nomenclatura: \emph{parola}\glsfirstoccur;
	\item i termini in lingua straniera o facenti parti del gergo tecnico sono evidenziati con il carattere \emph{corsivo}.
\end{itemize}