% !TEX encoding = UTF-8
% !TEX TS-program = pdflatex
% !TEX root = ../tesi.tex

%**************************************************************
\chapter{Introduzione}\label{ch:introduzione}
%**************************************************************

In questa sezione viene fatta una breve presentazione dell'azienda Imola Informatica S.p.A e una breve introduzione alle idee dello stage.
%\noindent Esempio di utilizzo di un termine nel glossario \\
%\gls{api}. \\
%
%\noindent Esempio di citazione in linea \\
%\cite{site:agile-manifesto}. \\
%
%\noindent Esempio di citazione nel pie' di pagina \\
%citazione\footcite{womak:lean-thinking} \\

%**************************************************************
\section{L'azienda}\label{sec:l'azienda}

Imola Informatica è una società indipendente di consulenza IT, entra in gioco ogni volta in cui una azienda pubblica o privata vuole migliorare i propri servizi. Innovare i propri processi di lavoro, gli approcci di management per cogliere le opportunità business offerte dalla trasformazione digitale.

È a servizio dei principali gruppi finanziari e assicurativi e ogni giorno è a fianco di grandi aziende e piccole startup nel gestire il cambiamento tecnologico e culturale.

Fa parte di una rete che condivide l’idea di fare innovazione a misura delle persone, delle imprese e della collettività. Sono impegnati per lo sviluppo consapevole di comunità locali e smart cities.

%**************************************************************
\section{L'idea}\label{sec:l'idea}

Le applicazioni mobili rappresentano oggi una delle principali sfide per la sicurezza informatica: negli ultimi anni se la rapida diffusione degli smartphone ha visto emergere il mobile come uno dei principali canali per l'erogazione di servizi, sono aumentati esponenzialmente gli attacchi contro le piattaforme mobili.\\
Le principali minacce del mondo delle applicazioni mobile sono:
\begin{itemize}
    \item \textit{Improper platform usage:} questa categoria di rischi comprende l'uso scorretto delle funzionalità offerte dalle piattaforme o il fallimento nell'utilizzo dei controlli di sicurezza offerti dalle piattaforme.
    \item \textit{Insecure data storage:} le minacce possono essere una di queste:
    \begin{itemize}
        \item un malintenzionato è riuscito a entrare in possesso di un dispositivo in modo illecito;
        \item un malware o un'applicazione ricompilato esegue del codice non sicuro;
    \end{itemize}
    Le conseguenze di questo tipo di minacce possono essere: il furto d'identità, violazione della privacy e frode.
    \item \textit{Insecure communication:}
    Nella progettazione di un'applicazione mobile, i dati sono spesso scambiati in un'archietettura client/server.
    Quando i dati devono essere inviati, l'attaccante potrebbe sfruttare le vulnerabilità per intercettare i dati sensibili.
    \item \textit{Insecure authentication:}
    Gli agenti di minaccia che sfruttano le vulnerabilità di autenticazione in genere lo fanno attraverso attacchi automatizzati che utilizzano strumenti disponibili o personalizzati.
    \item \textit{Insufficient cryptography:} gli agenti delle minacce includono: chiunque abbia accesso fisico ai dati che sono stati crittografati in modo improprio o malware mobile che agisce per conto di un avversario.
    Questa vulnerabilità comporta il recupero non autorizzato d'informazioni sensibili dal dispositivo mobile e violazione di privacy.
    \item \textit{Insecure authorization:}
    gli agenti delle minacce che sfruttano le vulnerabilità delle autorizzazioni in genere lo fanno attraverso gli attacchi automatici che utilizzano strumenti disponibili o personalizzati.
    \item \textit{Client code quality:} include entità che possono trasmettere input non attendibili alle chiamate di metodo effettuate all'interno del codice mobile. Questi tipi di problemi non sono necessariamente problemi di sicurezza in sé, ma possono causare problemi si vulnerabilità.
    \item \textit{Code tampering:}
    In genere, un utente malintenzionato sfrutta la modifica del codice tramite forme dannose delle app ospitante negli app store di terze parti.
    L'utente malintenzionato può anche indurre l'utente a installare l'app tramite attacchi di phishing. In genere, per sfruttare questo tipo di vulnerabilità, un utente malintenzionato eseguirà le seguenti operazioni: appoartare modifiche binarie dirette al file binario principale del pacchetto dell'applicazione, apportare modifiche binarie dirette alle risorse, e reindirizzare o sostituire le API di sistema per intercettare ed eseguire il codice esterno dannoso.

    \item \textit{Reverse engineering:}
    Un utente malintenzionato in genere scarica l'app di destinazione da un app store e la analizza nel proprio ambiente locale utilizzando una suite di strumenti diversi. Un malintenzionato deve eseguire un'analisi del file binario core finale per determinare la tabella di stringhe originale, il codice sorgente, le librerie, gli algoritmi e le risorse incorporate nell'app.
    \item \textit{Extraneous Functionality:}
    un utente malintenzionato cerca di comprendere le funzionalità estranee all'interno di un'app mobile  al fine di scoprire funzionalità nascoste nei sistemi di back-end.
    L'attaccante in genere sfrutta funzionalità estranee direttamente dai propri sistemi senza alcun coinvolgimento da parte degli utenti finali.
\end{itemize}
Le applicazioni mobili soffrono spesso di debolezze intrinseche dovute al design dell'applicazione e al suo sviluppo, debolezze che possono prevedere la memorizzazione d'informazioni sensibili sul device, possibilità di modificare l'applicazione (mediante \gls{repackaging} dell'App), o la possibilità di un semplice reverse engineering.

%**************************************************************
\section{Organizzazione del testo}\label{sec:organizzazione-del-testo}

\begin{description}
    \item[{\hyperref[ch:introduzione]{Il primo capitolo}}] dà un'introduzione generale all'azienda ospitante e l'idea fondante del progetto di stage;

    \item[{\hyperref[ch:descrizione-stage]{Il secondo capitolo}}] approfondisce gli obiettivi del progetto con i relativi requisiti e pianificazione;

    \item[{\hyperref[ch:analisi-requisiti]{Il terzo capitolo}}] approfondisce l'analisi dei requisiti del tool;

    \item[{\hyperref[ch:progettazione-e-codifica]{Il quarto capitolo}}] approfondisce la progettazione e la codifica del tool;
    
    \item[{\hyperref[ch:verifica-validazione]{Il quinto capitolo}}] approfondisce la verifica e la validazione del tool.
    
    \item[{\hyperref[ch:conclusioni]{Nel sesto capitolo}}] descrive le opinioni personali del laureando.
\end{description}

Riguardo la stesura del testo, relativamente al documento sono state adottate le seguenti convenzioni tipografiche:
\begin{itemize}
	\item gli acronimi, le abbreviazioni e i termini ambigui o di uso non comune menzionati vengono definiti nel glossario, situato alla fine del presente documento;
	\item per la prima occorrenza dei termini riportati nel glossario viene utilizzata la seguente nomenclatura: \emph{parola}\glsfirstoccur;
	\item i termini in lingua straniera o facenti parti del gergo tecnico sono evidenziati con il carattere \emph{corsivo}.
\end{itemize}