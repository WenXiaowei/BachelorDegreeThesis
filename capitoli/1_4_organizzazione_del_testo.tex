% !TEX encoding = UTF-8
% !TEX TS-program = pdflatex
% !TEX root = ../tesi.tex

\section{Organizzazione del testo}\label{sec:organizzazione-del-testo}
\begin{description}
    \item[{\hyperref[ch:introduzione]{Il primo capitolo}}] dà un'introduzione generale all'azienda ospitante e l'idea fondante del progetto di stage;

    \item[{\hyperref[ch:descrizione-stage]{Il secondo capitolo}}] approfondisce gli obiettivi del progetto con i relativi requisiti e pianificazione;

    \item[{\hyperref[ch:analisi-requisiti]{Il terzo capitolo}}] approfondisce l'analisi dei requisiti del tool;

    \item[{\hyperref[ch:progettazione-e-codifica]{Il quarto capitolo}}] approfondisce la progettazione e la codifica del tool;

    \item[{\hyperref[ch:verifica-validazione]{Il quinto capitolo}}] approfondisce la verifica e la validazione del tool.

    \item[{\hyperref[ch:conclusioni]{Nel sesto capitolo}}] descrive le opinioni personali del laureando.
\end{description}

Riguardo la stesura del testo, relativamente al documento sono state adottate le seguenti convenzioni tipografiche:
\begin{itemize}
	\item gli acronimi, le abbreviazioni e i termini ambigui o di uso non comune menzionati vengono definiti nel glossario, situato alla fine del presente documento;
	\item per la prima occorrenza dei termini riportati nel glossario viene utilizzata la seguente nomenclatura: \emph{parola}\glsfirstoccur;
	\item i termini in lingua straniera o facenti parti del gergo tecnico sono evidenziati con il carattere \emph{corsivo}.
\end{itemize}
