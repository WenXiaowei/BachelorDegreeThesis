% !TEX encoding = UTF-8
% !TEX TS-program = pdflatex
% !TEX root = ../tesi.tex

%**************************************************************
\chapter{Progettazione e codifica}
\label{ch:progettazione-e-codifica}
%**************************************************************

\intro{In questo capitolo vengono presentate le tecnologie e gli strumenti utilizzati, il ciclo di sviluppo del software adottato e i design pattern utilizzati.}\\

%**************************************************************
\section{Tecnologie e strumenti}
\label{sec:tecnologie-strumenti}

Di seguito viene data una panoramica delle tecnologie e strumenti utilizzati.

\subsection*{CFR - decompilatore java}
CFR è il decompilatore utilizzato per trasformare il codice bytecode \textit{.class} in \textit{.java};
\url{http://www.benf.org/other/cfr/index.html}

\subsection*{Apk Tool}
Apktool è un tool per effettuare il reverse engineering delle applicazioni Android. Può decodificare le risorse contenute nell'APK in una forma quasi uguale a quelli originali, può anche di ricostruire l'APK dopo le modifiche alle risorse modificate.
\url{https://ibotpeaches.github.io/Apktool/}
\subsection*{Dex2Jar}
Questo strumento permette di trasformare i file \textit{.dex} in formato \textit{.jar} in modo che possa essere letta attraverso un software di compressione, come 7Zip, per poter accedere quindi ai file di tipo \textit{.class}.
\url{https://github.com/pxb1988/dex2jar}

\subsection*{Android Emulator}
Lo strumento che permette di avviare degli emulatori Android, conosciuti anche come AVD. \`{E} stato utilizzato per eseguire l'applicazione ricompilata, per poter fare il dump dei dati presenti nell'area di storage dell'applicazione, come i file JSON, XML e SQLite, o come il file delle attività di rete.

\subsection*{Intellij IDEA}
L'IDE utilizzato per la scrittura del codice in Java.

\subsection*{Maven}
\subsection*{Astah}

\subsection*{Java}
\subsection*{Json}


%**************************************************************
\section{Ciclo di vita del software}
\label{sec:ciclo-vita-software}

%**************************************************************
\section{Progettazione}
\label{sec:progettazione}

\subsection{Package}\label{subsec:package}
\begin{namespacedesc}
    \classdesc{ApatLauncher}{}
    \classdesc{Utils}{}
\end{namespacedesc}

\subsubsection{Package  it.imolinfo.apat.controller} %**************************
Descrizione namespace 1
\begin{namespacedesc}
    \classdesc{AndroidManifestEditor.java}{\`{E} la classe che si occupa, dato un oggetto di tipo File Android Manifest, di modificare e/o aggiungere il tag \textit{debugabble} impostando il suo valore a \textit{true};}
    \classdesc{Controller}{Descrizione classe 2}
    \classdesc{MVCModule}{Descrizione classe 2}
\end{namespacedesc}

\subsubsection{Package  it.imolinfo.apat.model} %**************************
\begin{namespacedesc}
    \classdesc{Modello}{Descrizione classe 1}
    \classdesc{ModelState}{Descrizione classe 2}
    \classdesc{PDFWriter}{Descrizione classe 2}
    \classdesc{Result}{Descrizione classe 2}
    \classdesc{Unzipper}{Descrizione classe 2}
\end{namespacedesc}

\subsubsection{Package  it.imolinfo.apat.pattern} %**************************
\begin{namespacedesc}
    \classdesc{analyzer.Analyze}{Descrizione classe 1}
    \classdesc{analyzer.BaseAnalyze}{Descrizione classe 1}
    \classdesc{analyzer.BaseAnalyzeDecorator}{Descrizione classe 1}
    \classdesc{analyzer.DumpedFilesAnalyzer}{Descrizione classe 1}
    \classdesc{analyzer.FindDataBase}{Descrizione classe 1}
    \classdesc{analyzer.LambdaCounter}{Descrizione classe 1}
    \classdesc{analyzer.StringFinder}{Descrizione classe 1}
\end{namespacedesc}
\begin{namespacedesc}
    \classdesc{observer.Observable}{observable}
    \classdesc{observer.Observer}{observer}
\end{namespacedesc}
\begin{namespacedesc}
    \classdesc{cliCommandFactory.CommandFactory}{observable}
    \classdesc{cliCommandFactory.Commands}{observable}
    \classdesc{cliCommandFactory.UnixCommandFactory}{observable}
    \classdesc{cliCommandFactory.WindowsCommandFactory}{observable}
\end{namespacedesc}

\subsubsection{Package  it.imolinfo.apat.view} %**************************
\begin{namespacedesc}
    \classdesc{AnalysisChooser}{Descrizione classe 1}
    \classdesc{ApkFilter}{Descrizione classe 1}
    \classdesc{KeyStoreFilter}{Descrizione classe 1}
    \classdesc{TextFilter}{Descrizione classe 1}
    \classdesc{View}{Descrizione classe 1}
    \classdesc{WaitAction}{Descrizione classe 1}
\end{namespacedesc}


%**************************************************************
\section{Design Pattern utilizzati}\label{sec:design-pattern-utilizzati}
\subsection{Model View Controller}\label{subsec:model-view-controller}
\subsection{Observer}\label{subsec:observer}
\subsection{Decorator}\label{subsec:decorator}
%**************************************************************
\section{Codifica}\label{sec:codifica}
