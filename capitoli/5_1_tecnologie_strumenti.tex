% !TEX encoding = UTF-8
% !TEX TS-program = pdflatex
% !TEX root = ../tesi.tex

\section{Tecnologie}\label{sec:tecnologie}
Di seguito viene data una panoramica delle tecnologie e degli strumenti utilizzati.

\subsection*{CFR - decompilatore java}
CFR è il decompilatore utilizzato per trasformare il codice bytecode \textit{.class} in codice sorgente \textit{.java};
\url{http://www.benf.org/other/cfr/index.html}

\subsection*{Apk Tool}
Apktool è un tool per effettuare il reverse engineering delle applicazioni Android. Può decodificare le risorse contenute nell'APK in una forma quasi uguale a quelli originali, può anche di ricostruire l'APK dopo le modifiche alle risorse modificate.
\url{https://ibotpeaches.github.io/Apktool/}
\subsection*{Dex2Jar}
Questo strumento permette di trasformare i file \textit{.dex} in formato \textit{.jar} in modo che possa essere letta attraverso un software di compressione, come 7Zip, per poter accedere quindi ai file di tipo \textit{.class}.
\url{https://github.com/pxb1988/dex2jar}


\subsection*{Java}
In informatica, Java è un linguaggio di programmazione ad alto livello, orientato agli oggetti e tipizzazione statica, che si appoggia sull'omonima piattaforma software di esecuzione, specificamente progettato per essere il più possibile indipendente dalla piattaforma hardware di esecuzione.
\subsection*{Json}
JSON (JavaScript Object Notation) è un semplice formato per lo scambio di dati.
Per le persone è facile da leggere e scrivere, mentre per le macchine risulta facile da generare e analizzarne la sintassi.
JSON è basato su due strutture:
\begin{itemize}
    \item un insieme di coppie nome/valore, in diversi linguaggi, questo è realizzato come un oggetto, un record, uno struct, un dizionario, una tabella hash, un elenco di chiavi o un array associativo.
    \item un elenco ordinato di valori, nella maggior parte dei linguaggi questo si realizza con un array, un vettore o una sequenza.
\end{itemize}
\url{https://www.json.org/json-it.html}

\subsection*{XPath}\label{subsec:xpath}
In informatica XPath è un linguaggio, parte della famiglia XML, che permette di individuare i nodi all'interno di un documento XML. Le espressioni XPath, a differenza delle espressioni XML, non servono a identificare la struttura di un documento, bensì a localizzarne con precisione i nodi.


\section{Strumenti}\label{sec:strumenti}

\subsection*{Intellij IDEA}
L'IDE utilizzato per la scrittura del codice in Java sviluppato da JetBrains.
\url{https://www.jetbrains.com/idea/}

\subsection*{Android Emulator}
Lo strumento che permette di avviare degli emulatori Android, conosciuti anche come AVD. È stato utilizzato per eseguire l'applicazione ricompilata, per poter fare il dump dei dati presenti nell'area di storage dell'applicazione, come i file JSON, XML e SQLite, o come il file delle attività di rete.
\url{https://developer.android.com/studio/run/emulator}

\subsection*{Maven}
Maven è un tool di build automation, basato sul concetto di Project Object Model (POM), maven può gestire i build, fare il report e la documentazione da un singolo centro di informazione.
\url{https://maven.apache.org/}

\subsection*{Astah}
Astah è uno strumento di modellazione UML. Permette di creare vari diagrammi UML, come quelli di classe, di package, di sequenza e di attività.
\url{https://astah.net/products/astah-community/}

\subsection*{Git - GitLab}
Gitlab è una piattaforma web open-source che permette la gestione di repository e di funzioni d'issue tracking system.
Questa è un'istanza privata dell'azienda Imola Informatica S.p.A.
\url{https://git.imolinfo.it/users/sign_in}