% !TEX encoding = UTF-8
% !TEX TS-program = pdflatex
% !TEX root = ../tesi.tex

%**************************************************************


\section{Consuntivo finale}\label{sec:consuntivo-finale}
Il lavoro è stato svolto nei tempi previsti, il numero di ore pianificato per ognuno delle fasi dello sviluppo sono stati sufficienti.
Nella seguente tabella sono riportate le attività svolte con le ore previste e tra parentesi quelle effettive.

\begin{longtable}{|C{3cm}|C{8cm}|C{2cm}|}
    \hline
    \textbf{Attività} &
    \textbf{Descrizione} &
    \textbf{Previste} (\textbf{Effettive}) \\\hline
    Formazione     & Formazione sulle tecnologie.
    & 40 (35) \\\hline
    Pianificazione & Pianificazione delle attività da svolgere.
    & 40 (38)    \\\hline
    Analisi dei requisiti & Individuazione dei casi d'uso, dei requisiti e creazione delle tabelle per il tracciamento dei requisiti.
    & 40 (37) \\\hline

    Progettazione & Progettazione del tool APAT includendo:
    \begin{itemize}\itemsep0em
        \item progettazione della logica per creazione dei comandi;
        \item progettazione della logica dell'esecuzione dei comandi generati;
        \item progettazione della logica per effettuare l'analisi del codice sorgente;
        \item progettazione dell'interfaccia grafica.
    \end{itemize}
    &60 (50)\\\hline
    Codifica & Codifica del tool APAT includendo:
    \begin{itemize}\itemsep0em
        \item codifica della logica per creazione dei comandi;
        \item codifica della logica dell'esecuzione dei comandi generati;
        \item codifica della logica per effettuare l'analisi del codice sorgente;
        \item codifica dell'interfaccia grafica.
    \end{itemize}
    & 100 (118) \\\hline
    Test           & Analisi statica, stesura dei test automatici, test manuali del tool.
    & 20 (22)   \\\hline
    Collaudo       & User Acceptance Test, verifica soddisfacimento requisiti.
    & 20 (20)    \\\hline
    \caption{Attività svolte}
\end{longtable}
Le pianificazioni sono state sufficientemente precise, in questo modo, le attività sono state svolte nei tempi previsti della durata dello stage.
Nonostante si fossero presentate alcune difficoltà la pianificazione non ha subito gravi ritardi grazie all'aiuto del tutor aziendale.