% !TEX encoding = UTF-8
% !TEX TS-program = pdflatex
% !TEX root = ../tesi.tex
\section{L'idea}\label{sec:l'idea}
Le applicazioni mobili rappresentano oggi una delle principali sfide per la sicurezza informatica.
La rapida diffusione degli smartphone, infatti, ha visto emergere il settore mobile come uno dei principali canali per l'erogazione di servizi provocando, di conseguenza, un aumento esponenziale degli attacchi contro tali piattaforme.
Tra le minacce\footcite{site:owasp} fondamentali, distinguiamo:

\begin{itemize}
    \item \textit{Improper platform usage:} questa categoria di rischi comprende l'uso scorretto delle funzionalità offerte dalle piattaforme o il fallimento nell'utilizzo dei controlli di sicurezza offerti dalle piattaforme.
    \item \textit{Insecure data storage:} le minacce possono essere una delle seguenti:
    \begin{itemize}
        \item un malintenzionato è riuscito a entrare in possesso di un dispositivo in modo illecito;
        \item un malware o un'applicazione ricompilata eseguono del codice non sicuro;
    \end{itemize}
    Le conseguenze di questo tipo di minacce possono essere: il furto d'identità, la violazione della privacy e frode.
    \item \textit{Insecure communication:}
    nella progettazione di un'applicazione mobile, i dati sono spesso scambiati in un'architettura client/server.
    Quando i dati devono essere inviati, l'attaccante potrebbe sfruttare le vulnerabilità per intercettare i dati sensibili.
    \item \textit{Insecure authentication:}
    gli agenti di minaccia che sfruttano le vulnerabilità di autenticazione in genere lo fanno attraverso attacchi automatizzati che utilizzano strumenti disponibili o personalizzati.
    \item \textit{Insufficient cryptography:} gli agenti delle minacce includono: chiunque abbia accesso fisico ai dati che sono stati crittografati in modo improprio o malware mobile che agisce per conto di un avversario.
    Questa vulnerabilità comporta il recupero non autorizzato d'informazioni sensibili dal dispositivo mobile e violazione della privacy.
    \item \textit{Insecure authorization:}
    gli agenti delle minacce che sfruttano le vulnerabilità delle autorizzazioni in genere lo fanno attraverso gli attacchi automatici che utilizzano strumenti disponibili o personalizzati.
    \item \textit{Client code quality:} include le entità che possono trasmettere in input dati non attendibili alle chiamate di metodo effettuate all'interno del codice mobile.
    Tale categoria di problemi non rappresenta in sè una grave problematica di sicurezza, ma potrebbe provocare vulnerabilità indesiderate.

    \item \textit{Code tampering:}
    un utente malintenzionato sfrutta la modifica del codice tramite forme dannose delle applicazioni ospitante negli app store delle terze parti.
    L'utente malintenzionato può anche indurre l'utente a installare l'app tramite attacchi di phishing.\\
    In genere, per sfruttare questo tipo di vulnerabilità, un utente malintenzionato eseguirà le seguenti operazioni: apportare modifiche direttamente al file binario principale del pacchetto dell'app ospitante, apportare modifiche binarie dirette alle risorse, e reindirizzare o sostituire le API di sistema per intercettare ed eseguire il codice esterno dannoso.

    \item \textit{Reverse engineering:}
    un utente malintenzionato in genere scarica l'app di destinazione da un app store e la analizza nel proprio ambiente locale utilizzando una suite di strumenti.
    Un malintenzionato deve eseguire un'analisi del file binario core finale per determinare la tabella di stringhe originale, il codice sorgente, le librerie, gli algoritmi e le risorse incorporate nell'app.
    \item \textit{Extraneous Functionality:}
    un utente malintenzionato cerca di comprendere le funzionalità estranee all'interno di un'app mobile  al fine di scoprire funzionalità nascoste nei sistemi di back-end.
    L'attaccante in genere sfrutta funzionalità estranee direttamente dai propri sistemi senza alcun coinvolgimento da parte degli utenti finali.
\end{itemize}
Le applicazioni mobili soffrono spesso di debolezze intrinseche dovute al design dell'applicazione e al suo sviluppo, debolezze che possono prevedere la memorizzazione d'informazioni sensibili sul device, possibilità di modificare l'applicazione (mediante \gls{repackaging} dell'app), o la possibilità di un semplice \gls{reverse_engineering}.

La sicurezza è diventata un fattore critico sulle piattaforme mobile.
Tradizionalmente, essa viene declinata secondo la cosiddetta triade CIA\footcite{samonas2014cia} che, in italiano, è possibile tradurre come Confidenzialità, Integrità e Disponibilità (dall'inglese Availability) di un sistema.
Sui dispositivi mobile garantire il rispetto di tali proprietà è assai più complesso rispetto ai sistemi tradizionali, per via della loro eterogeneità, diffusione e distribuzione, per questi motivi è necessario ripensare alla sicurezza dei sistemi per garantire la protezione degli utenti, delle loro risorse e delle piattaforme.

La sicurezza diviene sempre più un fattore fondamentale da prevedere in tutte le fasi del ciclo di vita del software e non relegata alle sole fasi post rilascio: quindi è fondamentale ridefinire il concetto di Software Development Lifecycle in Secure Software Development Lifecycle\footcite{site:ssdl}, con una connotazione più orientata alla sicurezza in tutte le sue fasi, ovvero:
\begin{itemize}
    \item \textit{Analisi:} durante l'analisi bisogna individuare quali sono i rischi che il software è sottoposto, quindi individuare i requisiti che il software deve soddisfare per contrastare i potenziali attacchi;
    \item \textit{Progettazione:} nel corso della fase di progettazione è necessario prestare attenzione nell'adozione delle librerie esterne, assicurandosi che quest'ultime non abbiano falle di sicurezza che minano l'integrità dell'intero software.
    Inoltre, è necessario adottare i standard di sicurezza consolidati;
    \item \textit{Sviluppo:} in questa fase le azioni\footcite{site:security-best-practice} che si possono adottare sono molteplici, per esempio:
    \begin{itemize}
        \item validare gli input;
        \item controllare l'accesso;
        \item evitare le chiavi hardcoded;
        \item evitare di salvare le password in chiaro;
        \item gestire gli utenti, le sessioni e i permessi;
        \item standardizzare la documentazione in modo da facilitare la manutenzione del codice;
        \item adottare la pratica del code review;
        \item adottare il principio \gls{kiss};
        \item adottare la pratica del \gls{sast} e \gls{sca}.
    \end{itemize}
    \item \textit{Test:} durante la fase di test è necessario invece adottare la \gls{dast};
    \item \textit{Deployment e Maintenance:} si tratta di messa in produzione del software sviluppato, quindi bisogna effettuare l'analisi della sicurezza e configurare sia il software che l'hardware sul quale è in esecuzione in modo corretto.
\end{itemize}

