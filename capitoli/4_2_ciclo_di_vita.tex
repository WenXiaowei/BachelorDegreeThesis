% !TEX encoding = UTF-8
% !TEX TS-program = pdflatex
% !TEX root = ../tesi.tex


\section{Ciclo di vita del software}\label{sec:ciclo-vita-software}
Il modello adottato per lo sviluppo del progetto è il modello incrementale\footcite{womak:ingegneria-software}.
Esso permette di suddividere la durata del progetto in diversi incrementi, in ognuno dei quali sono stati definiti degli obiettivi/requisiti da raggiungere.
Inoltre, tale modello di sviluppo offre i seguenti vantaggi:
\begin{itemize}
    \item assegnazione delle diverse priorità agli obiettivi da raggiungere, in modo che quelli più importanti vengano raggiunti prima;
    \item dimostrazione del prodotto al termine di ogni incremento in questo modo si assicura che il risultato di ogni incremento sia consono alle aspettative del cliente;
    \item verifica del prodotto avviene a ogni incremento in modo da garantire la correttezza del prodotto di ogni incremento;
    \item in ogni incremento vengono aggiunte delle funzionalità corrette e funzionanti, allora il prodotto finale soddisfa i requisiti del cliente;
\end{itemize}