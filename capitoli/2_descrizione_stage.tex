% !TEX encoding = UTF-8
% !TEX TS-program = pdflatex
% !TEX root = ../tesi.tex

%**************************************************************
\chapter{Descrizione dello stage}
\label{ch:descrizione-stage}
%**************************************************************

\intro{Questo capitolo contiene l'introduzione del progetto, l'analisi dei rischi, la specifica degli obiettivi e la pianificazione dei lavori.}\\

%**************************************************************

%**************************************************************
\section{Analisi preventiva dei rischi}\label{sec:analisi-preventiva-dei-rischi}

Durante la fase di analisi iniziale sono stati individuati alcuni possibili rischi a cui si potrà andare incontro.
Si è quindi proceduto a elaborare delle possibili soluzioni per far fronte a tali rischi.\\

\begin{risk}{Tecnologie/framework sconosciute}
    \riskdescription{Durante lo svolgimento dello stage, sono richieste l'uso di alcune tecnologie o di framework fin'ora mai viste dallo stagista.}
    \risksolution{È stato programmato un periodo di auto-formazione riguardanti le tecnologie che si prevede di utilizzare, inoltre, il tutor aziendale si è reso disponibile di aiutare lo stagista se necessario}
    \label{risk:tecnologie-framework-sconosciute}
\end{risk}
\begin{risk}{Guasti hardware}
    \riskdescription{È possibile che il computer assegnato possa incorrere in guasti hardware, rischiando così di causare rallentamenti o addirittura perdita del lavoro.}
    \risksolution{Il lavoro svolto verrà versionato nel sistema di versionamento dell'azienda.}
    \label{risk:guasti-hardware}
\end{risk}
\begin{risk}{Incomprensioni o scelte non ottimali}
    \riskdescription{A causa dell'inesperienza può accadere che l'attività da svolgere siano fraintese o valutate erroneamente, causando la realizzazione di un prodotto non consono alle aspettative dell'azienda.}
    \risksolution{Quando ci sono delle scelte importanti, la scelta viene fatta in concordanza con il tutor aziendale.}
    \label{risk:incomprensioni-o-scelte-non-ottimali}
\end{risk}
%\begin{risk}{}
%    \riskdescription{}
%    \risksolution{}
%    \label{risk:}
%\end{risk}

%**************************************************************
\section{Requisiti e obiettivi}\label{sec:requisiti-e-obiettivi}
Lo scopo dello stage è la realizzazione di un tool di analisi di applicazioni Android che permetta di automatizzare operazioni di reverse engineering dei pacchetti delle app mobile (APK), effettuando nell'ordine:
\begin{enumerate}
    \setlength\itemsep{0.1em}
    \item decompilazione sorgenti;
    \item analisi dei file sorgenti ottenuti dalla decompilazione;
    \item generazione di un report dell'analisi;
    \item creazione di un file \gls{pcap} che contiene i dettagli delle attività di rete;
    \item repackaging dell'applicativo;
    \item firma dell'APK ottenuto dal repackaging.
\end{enumerate}

Al termine dell'esecuzione dovranno essere rese disponibili informazioni quali:
\begin{itemize}
    \setlength\itemsep{0.1em}
    \item sorgenti decompilati;
    \item stringhe estratte dai sorgenti per l'individuazione di chiavi hard coded;
    \item contenuto della storage area dell'app;
    \item stringhe estratte dalla storage area dell'app per l'inviduazione d'informazioni sensibili;
    \item file di trace per le operazioni di rete effettuate in un file cap.
\end{itemize}

%**************************************************************
\section{Pianificazione}\label{sec:pianificazione}
\begin{tabularx}{\textwidth}{|c|X|}
    \hline
    \textbf{Durata in ore} & \textbf{Descrizione dell'attività} \\\hline
    40 & Studio e approfondimento delle tecnologie di sviluppo \\\hline
    40 & Analisi dei requisiti \\\hline
    120 & Stesura del codice \\\hline
    80 & Collaudo e risoluzione bug \\\hline
    40 & Analisi delle performance e assestamento dei risultati \\\hline
    \textbf{Totale ore} & \multicolumn{1}{|c|}{\textbf{320}} \\\hline
\end{tabularx}