\section{Test di sistema}\label{sec:test-di-sistema}
In ingegneria del software il test di sistema è un procedimento, parte del ciclo di vita del software, utilizzato per individuare le carenze di correttezza, completezza e affidabilità delle componenti software in corso di sviluppo.
Consiste nell'esecuzione del software da parte del collaudatore per verificare le funzionalità offerte del software in relazione con i requisiti.
\setcounter{rowcount}{0}

\subsection{Specifica dei test}\label{subsec:specifica-dei-test-sistema}
Di seguito sono riportati i test d'integrazione che verificano il corretto funzionamento del sistema.
\begin{center}
    \begin{longtable}{ | C{2.5cm} |C{2cm} |C{7cm} |C{1.5cm}|}
        \hline
        \textbf{Identificativo} &
        \textbf{Requisito} &
        \textbf{Descrizione} &
        \textbf{Stato} \\\hline

        \idTest{TS} & R-1-F-O    &Verificare che il tool permetta di selezionare solo i file APK. & I \\\hline
        \idTest{TS} & R-1.1-F-O  &Verificare che venga mostrato un messaggio di errore quando il file selezionato non è APK. & I \\\hline
        \idTest{TS} & R-2-F-O    &Verificare che il file APK venga decompilato correttamente.& I \\\hline
        \idTest{TS} & R-2.1-F-O  &Verificare che venga mostrato un messaggio quando l'APK è stato decompilato correttamente.& I \\\hline
        \idTest{TS} & R-2.2-F-O  &Verificare che nel file AndroidManifest.xml venga aggiunto l'attributo debuggable correttamente.& I \\\hline
        \idTest{TS} & R-3-F-O    &Verificare che venga mostrato il messaggio d'errore quando la decompilazione non è andato a buon fine.& I \\\hline
        \idTest{TS} & R-4-F-O    &Verificare che l'applicazione ricompilato venga installata correttamente nell'AVD selezionato.& I \\\hline
        \idTest{TS} & R-4.1-F-D  &Verificare che l'APK venga ricompilata correttamente.& I \\\hline
        \idTest{TS} & R-4.2-F-D  &Verificare che venga avviata l'applicazione.& I \\\hline
        \idTest{TS} & R-4.3-F-O  &Verificare che il tool permetta di selezionare un AVD presente nel sistema.& I \\\hline
        \idTest{TS} & R-4.4-F-O  &Verificare che venga visualizzata l'elenco degli AVD presenti nel sistema.& I \\\hline
        \idTest{TS} & R-4.5-F-O  &Verificare che il tool permetta di confermare la selezione dell'AVD.& I \\\hline
        \idTest{TS} & R-5-F-O    &Verificare che venga mostrato un messaggio quando non sono stati rilevati nessun AVD nel sistema.& I \\\hline
        \idTest{TS} & R-6-F-O    &Verificare che deve mostrare un messaggio d'errore quando l'AVD non è stato avviato correttamente.& I \\\hline
        \idTest{TS} & R-7-F-O    &Verificare che vengano effettivamente scaricati i file dall'area di storage dell'applicazione.& I \\\hline
        \idTest{TS} & R-7.1-F-O  &Verificare che sia permesso all'utente di selezionare il path dove collocare i file scaricati dall'area di storage.& I \\\hline
        \idTest{TS} & R-8-F-D    &Verificare che vengano decodificati i file dex.& I \\\hline
        \idTest{TS} & R-8.1-F-O  &Verificare che venga visualizzato il messaggio quando la decodifica dei file dex è avvenuto con successo.& I \\\hline
        \idTest{TS} & R-8.2-F-O  &Verificare che venga fatta selezionare i file dex da decodificare.& I \\\hline
        \idTest{TS} & R-8.3-F-O  &Verificare che confermata la selezione dei file dex.& I \\\hline
        \idTest{TS} & R-8.4-F-O  &Verificare che sia permessa la selezione dei path di destinazione dei file decodificati.& I \\\hline
        \idTest{TS} & R-9-F-O    &Verificare che venga mostrato il messaggio d'errore quando la decodificato dei file dex non è andato a buon fine.& I \\\hline
        \idTest{TS} & R-10-F-F   &Verificare che il tool permetta di firmare l'apk ricompilato.& I \\\hline
        \idTest{TS} & R-10.1-F-F &Verificare che il tool permetta di selezionare un file di tipo keystore.& I \\\hline
        \idTest{TS} & R-10.2-F-F &Verificare che venga mostrato un messaggio d'errore quando il file selezionato non è di tipo keystore.& I \\\hline
        \idTest{TS} & R-10.3-F-F &Verificare che il tool permetta l'inserimento dell'alias della chiave da utilizzare.& I \\\hline
        \idTest{TS} & R-10.4-F-F &Verificare che il tool permetta l'inserimento della password del keystore da utilizzare.& I \\\hline

        \idTest{TS} & R-11-F-F   &Verificare che venga mostrato il messaggio d'errore quando i dati inseriti non sono corretti.& I \\\hline
        \idTest{TS} & R-12-F-O   &Verificare che venga mostrato un messaggio quando il file selezionato non è valido.& I \\\hline
        \idTest{TS} & R-13-F-O   &Verificare che il tool permetta di decodificare i file dex in .java.& I \\\hline
        \idTest{TS} & R-14-F-O   &Verificare che il tool permetta di effettuare l'analisi del codice java.& I \\\hline
        \idTest{TS} & R-14.1-F-O &Verificare che il tool permetta di selezionare diversi tipi di analisi.& I \\\hline
        \idTest{TS} & R-14.2-F-O &Verificare che il tool permetta di selezionare la destinazione per il file di report.& I \\\hline
        \idTest{TS} & R-15-F-D   &Verificare che il tool permetta di avviare un AVD presente nel sistema.& I \\\hline
        \idTest{TS} & R-15.1-F-D &Verificare che il tool permetta di selezionare l'AVD presente nel sistema.& I \\\hline
        \idTest{TS} & R-15.2-F-D &Verificare che il tool permetta di visualizzare l'elenco degli AVD presenti nel sistema.& I \\\hline
        \idTest{TS} & R-16-F-D   &Verificare che il tool permetta d'inserire le opzioni di proxy.& I \\\hline
        \idTest{TS} & R-16.1-F-D &Verificare che il tool permetta d'inserire l'indirizzo del server proxy.& I \\\hline
        \idTest{TS} & R-16.2-F-D &Verificare che il tool permetta d'inserire il numero di porta del server proxy.& I \\\hline
        \idTest{TS} & R-17-F-D   &Verificare che il tool registri il traffico di rete.& I \\\hline
        \idTest{TS} & R-17.1-F-D &Verificare che il tool generi un file di report al termine delle operazioni effettuate.& I \\\hline

        \caption{Test di sistema}
    \end{longtable}
\end{center}
\setcounter{subCount}{0}
\setcounter{rowcount}{0}
