% !TEX encoding = UTF-8
% !TEX TS-program = pdflatex
% !TEX root = ../tesi.tex

\section{Requisiti e obiettivi}\label{sec:requisiti-e-obiettivi}

Lo scopo dello stage è la realizzazione di un tool di analisi di applicazioni Android che permetta di automatizzare operazioni di reverse engineering dei pacchetti delle app mobile (APK), effettuando nell'ordine:
\begin{enumerate}
    \setlength\itemsep{0.1em}
    \item decompilazione sorgenti;
    \item analisi dei file sorgenti ottenuti dalla decompilazione;
    \item generazione di un report dell'analisi;
    \item creazione di un file \gls{pcap} che contiene i dettagli delle attività di rete;
    \item repackaging dell'applicativo;
    \item firma dell'APK ottenuto dal repackaging.
\end{enumerate}

Al termine dell'esecuzione dovranno essere rese disponibili informazioni quali:
\begin{itemize}
    \setlength\itemsep{0.1em}
    \item sorgenti decompilati;
    \item stringhe estratte dai sorgenti per l'individuazione di chiavi hard coded;
    \item contenuto della storage area dell'app;
    \item stringhe estratte dalla storage area dell'app per l'inviduazione d'informazioni sensibili;
    \item file di trace per le operazioni di rete effettuate in un file cap.
\end{itemize}

%**************************************************************