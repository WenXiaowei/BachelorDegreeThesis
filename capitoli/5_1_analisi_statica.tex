\section{Analisi statica}\label{sec:analisi-statica}

L'analisi statica del codice\footcite{site:analisi_static} è l'analisi del software che è effettuata senza eseguire
il codice.
Si pone in contrasto con l'analisi dinamica che invece richiede l'esecuzione del
programma.
Il termine è spesso utilizzato per indicare l'analisi eseguita da tool
automatici che nel caso del coinvolgimento dell'essere umano diventa code review.
Essa può  essere di due tipi, \gls{walkthrough} e \gls{inspection}.
Il tool utilizzato per l'analisi statica automatizzata è un plugin di Maven
chiamato checkstyle\footcite{site:checkstyle} che è capace di generare un report indicando la presenza o
meno del codice che non è conforme alle convenzioni definite dal programmatore.
In questo progetto sono state definite le seguenti regole:

In questo progetto sono state definite le seguenti regole:
\begin{itemize}
    \item complessità delle espressioni booleane: massimo 3;
    \item \gls{ciclomatica} delle funzioni: 16;
    \item lunghezza massima di ogni riga: 150 caratteri;
    \item lunghezza massima di ogni metodo: 100 righe;
    \item lunghezza massima di ogni file: 1000 righe;
    \item presenza dei blocchi di catch vuoi: vietata.
\end{itemize}

Il risultato dell'analisi è stato positivo, poiché non presenta nessuna riga del codice che non rispetti le regole sopracitate.