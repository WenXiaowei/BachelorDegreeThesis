\section{Tracciamento dei requisiti}\label{sec:tracciamento-dei-requisiti}

\newcounter{rowcount}
\setcounter{rowcount}{0}
\newcounter{subCount}
\setcounter{subCount}{0}

\subsection{Classificazione}\label{subsec:classificazione}
Di seguito sono riportati i requisiti individuati durante l'attività di analisi.
Tali requisiti sono stati individuati dai casi d'uso e dai colloqui con il tutor aziendale \tutorAziendale.
I requisiti individuati sono stati divisi in:
\begin{itemize}
    \item \textbf{Requisiti funzionali:} insieme di requisiti che definiscono le azioni fondamentali che devono avvenire in grado di processare un input e di generare un output;
    \item \textbf{Requisiti dichiarativi:} insieme di requisiti che rappresentano un vincolo di natura realizzativa, normativa o contrattuale;
    \item \textbf{Requisiti qualitativi:} insieme di requisiti che garantiscono una certa qualità al prodotto e che indicano le best practice usate per la realizzazione.
\end{itemize}
Inoltre, a ogni requisito è stata assegnata un'importanza:
\begin{itemize}
    \item \textbf{Obbligatori:} requisito al quale non si può rinunciare, indispensabile per il corretto funzionamento del prodotto;
    \item \textbf{Desiderabile:} requisito non necessario ma che porta valore aggiunto al prodotto;
    \item \textbf{Facoltativo:} requisito che risulta essere relativamente utile oppure contrattabile con il proponente in un momento successivo.
\end{itemize}

\subsection{Requisiti funzionali}\label{subsec:requisiti-funzionali}
\renewcommand{\arraystretch}{1.5}
\begin{center}
    \begin{longtable}{ | c| C{7cm} |C{3cm} |}
        \hline
        \textbf{Identificativo} & \textbf{Descrizione} &\textbf{Fonte}\\\hline
        \idRequesiti{F-O} & Il tool deve permettere di selezionare un file APK.& UC-1\\\hline
        \idRequesitiSub{F-O} & Il tool deve permettere di mostrare un messaggio di errore se file selezionato non è valido.& UC-1\\\hline
        \setcounter{subCount}{0}

        \idRequesiti{F-O} & Il tool deve permettere di avviare la decompilazione dell'APK selezionato.& UC-2\\\hline
        \idRequesitiSub{F-O} & Il tool deve permettere di visualizzare un messaggio di decompilazione avvenuto con successo.& UC-2\\\hline
        \idRequesitiSub{F-O} & Il tool deve aggiungere il flag android:debuggable="true" nel AndroidManifest.xml del decompilato.& UC-2\\\hline
        \setcounter{subCount}{0}

        \idRequesiti{F-O} & Il tool deve permettere di visualizzare il messaggio di errore quando la decompilazione non è terminato con successo.& UC-3\\\hline
        \idRequesiti{F-O} & Il tool deve permettere d'installare l'APK decompilato e manomesso su un AVD. & UC-4 \\\hline
        \idRequesitiSub{F-D} & Il tool deve permettere di ricompilare l'APK decompilato.& UC-4\\\hline
        \idRequesitiSub{F-D} & Il tool deve permettere di avviare l'applicazione.& UC-4\\\hline
%        \setcounter{subCount}{0}

        \idRequesitiSub{F-O} & Il tool deve permettere di selezionare un'AVD presente nel sistema.& UC-4 \\\hline
        \idRequesitiSub{F-O} & Il tool deve permettere di visualizzare l'elenco delle AVD presenti nel sistema operativo.& UC-4\\\hline
        \idRequesitiSub{F-O} & Il tool deve permettere di selezionare un'AVD presente nel sistema operativo.& UC-4\\\hline
        \idRequesitiSub{F-O} & Il tool deve permettere di confermare la selezione dell'AVD.& UC-4\\\hline
        \setcounter{subCount}{0}

        \idRequesiti{F-O} &Il tool deve permettere di visualizzare il messaggio quando non sono stati rilevati nessun AVD. & UC-5 \\\hline
        \idRequesiti{F-O} &Il tool deve permettere di visualizzare il messaggio di errore se l'AVD selezionato non si è avviato correttamente.& UC-6 \\\hline
        \idRequesiti{F-O} &Il tool deve permettere di fare il dump dello storage interna dell'applicazione.& UC-7 \\\hline
        \idRequesitiSub{F-O} & Il tool deve permettere di selezionare il path dove collocare i dati copiati.& UC-7\\\hline
        \setcounter{subCount}{0}

        \idRequesiti{F-D} &Il tool deve permettere di visualizzare i dex ottenuti dalla decompilazione.& UC-8\\\hline
        \idRequesitiSub{F-O} & Il tool deve permettere di visualizzare il messaggio quando la decodifica è avvenuto con successo.& UC-8\\\hline
        \idRequesitiSub{F-O} & Il tool deve permettere di far selezionare il dex che l'utente vuole decodificare.& UC-8\\\hline
        \idRequesitiSub{F-O} & Il tool deve permettere di confermare la selezione del dex da decodificare.& UC-8\\\hline
        \idRequesitiSub{F-O} & Il tool deve permettere di far selezionare il path di dove salvare i file ottenuti dalla decodifica.& UC-8\\\hline
        \setcounter{subCount}{0}

        \idRequesiti{F-O} &Il tool deve permettere di visualizzare il messaggio d'errore se la decodifica non è andato a buon fine.& UC-9 \\\hline

        \idRequesiti{F-F}& Il tool deve permettere di firmare l'APK ricompilato.& UC-10 \\\hline
        \idRequesitiSub{F-F} & Il tool deve permettere di selezionare un file di tipo keystore.&UC-10\\\hline
        \idRequesitiSub{F-F}& Il tool deve permettere di mostrare un messaggio di errore quando viene selezionato un file non keystore.& UC-10 \\\hline
        \idRequesitiSub{F-F} & Il tool deve permettere d'inserire l'alias della chiave da utilizzare.& UC-10\\\hline
        \idRequesitiSub{F-F} & Il tool deve permettere d'inserire la password del keystore da utilizzare.& UC-10 \\\hline
        \setcounter{subCount}{0}

        \idRequesiti{F-F}& Il tool deve permettere di mostrare dei messaggi quando le credenziali inseriti non sono corretti.& UC-11 \\\hline
        \idRequesiti{F-O}& Il tool deve permettere di mostrare dei messaggi quando è stato selezionato un file non valido.& UC-12 \\\hline
        \idRequesiti{F-O}& Il tool deve permettere di decodificare i file \textit{.dex} in \textit{.java} & UC-13\\\hline
        \idRequesiti{F-O}& Il tool deve permettere di permettere di effettuare dell'analisi del codice java& UC-14 \\\hline
        \idRequesitiSub{F-O}& Il tool deve permettere di selezionare diverse opzioni di analisi &UC-14.1 \\\hline
        \idRequesitiSub{F-O}& Il tool deve permettere di selezionare il path dove collocare i risultati dell'analisi.& UC-14.2\\\hline
        \setcounter{subCount}{0}

        \idRequesiti{F-D} & Il tool deve permettere di avviare un'AVD presente nel sistema.&UC-15\\\hline
        \idRequesitiSub{F-D} & Il tool deve permettere di visualizzare l'elenco delle AVD presenti nel sistema operativo.&UC-15\\\hline
        \idRequesitiSub{F-D} & Il tool deve permettere di visualizzare di selezionare un'AVD presente nel sistema.&UC-15\\\hline
        \idRequesitiSub{F-D} & Il tool deve permettere di specificare se avviare l'AVD con opzione di proxy.&UC-15\\\hline
        \setcounter{subCount}{0}
        \idRequesiti{F-D} & Il tool deve permettere d'inserire le informazioni per il proxy.&UC-17\\\hline
        \idRequesitiSub{F-D}& Il tool deve permettere d'inserire l'indirizzo del server proxy.&UC-17\\\hline
        \idRequesitiSub{F-D}& Il tool deve permettere d'inserire il numero di porta del server proxy.&UC-17\\\hline
        \setcounter{subCount}{0}
        \idRequesiti{F-D} & Il tool deve permettere di registrare il traffico di rete.&UC-18\\\hline
        \idRequesitiSub{F-D}& Il tool deve permettere d'iniziare la registrazione del traffico di rete.&UC-18\\\hline
        \idRequesitiSub{F-D}& Il tool deve permettere di fermare la registrazione del traffico di rete.&UC-18\\\hline
        \idRequesitiSub{F-D}& Il tool deve generare il file che contiene i dettagli del traffico di rete.&UC-18\\\hline
%        \setcounter{subCount}{0}


%        \idRequesiti{F-D} &     \\\hline
%        \idRequesitiSub{F-D}&   \\\hline
        \caption{Requisiti funzionali}
    \end{longtable}
\end{center}
\setcounter{rowcount}{0}

\subsection{Requisiti di vincolo}\label{subsec:requisiti-vincolo}
\renewcommand{\arraystretch}{1.5}
\begin{center}
    \begin{longtable}{ | c| C{7cm} |C{3cm} |}
        \hline
        \textbf{Identificativo} & \textbf{Descrizione} &\textbf{Fonte}\\\hline
        \idRequesiti{V-D} &Il tool può essere un tool da righe di commando & Colloquio col tutor aziendale.\\\hline
        \idRequesiti{V-D} &Il tool può essere un tool dotato di GUI & Colloquio col tutor aziendale.\\\hline
        \idRequesiti{V-D} &Il tool può essere sviluppato in JAVA & Colloquio col tutor aziendale.\\\hline
        \idRequesiti{V-F} &Il tool può essere sviluppato utilizzando i linguaggi funzionali.& Colloquio col tutor aziendale.\\\hline
        \caption{Requisiti di vincolo}
    \end{longtable}
\end{center}
\setcounter{rowcount}{0}

\subsection{Requisiti qualitativi}\label{subsec:requisiti-qualitativi}
\renewcommand{\arraystretch}{1.5}
\begin{center}
    \begin{longtable}{ | c| C{7cm} |C{3cm} |}
        \hline
        \textbf{Identificativo} & \textbf{Descrizione} &\textbf{Fonte}\\\hline
        \idRequesiti{Q-O}&Il codice sorgente deve essere versionato col sistema di versionamento dell'azienda ospitante. &Colloquio col tutor aziendale \\\hline
        \idRequesiti{Q-O}&Il code coverage dei test unitari deve superare 80\%.& Colloquio col tutor aziendale\\\hline
        \idRequesiti{Q-D}&Il code coverage dei test unitari deve essere 100\%.& Colloquio col tutor aziendale\\\hline
        \idRequesiti{Q-O}&Il Il codice sorgente deve essere sotto licenza GPL v3.& Colloquio col tutor aziendale\\\hline
        \caption{Requisiti qualitativi}
    \end{longtable}
\end{center}
\setcounter{subCount}{0}
\setcounter{rowcount}{0}