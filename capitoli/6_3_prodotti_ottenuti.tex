% !TEX encoding = UTF-8
% !TEX TS-program = pdflatex
% !TEX root = ../tesi.tex
%**************************************************************

\section{Prodotti realizzati}\label{sec:prodotti-ottenuti}

Al termine dello stage, il tool APAT è composto da 3216 righe di codice in Java, suddivisi a loro volta in 52 file e dipende da due file si configurazione in formato JSON.\\
Per l'analisi dei requisiti sono stati realizzati sette diagrammi dei casi d'uso componendo così il documento \textit{Analisi dei requisiti}.
Per la progettazione del tool, sono stati realizzati sei diagrammi delle classi e dei package costituendo di conseguenza il documento \textit{Manuale dello sviluppatore}.
Lo strumento APAT è composto da quattro viste, quella principale in cui sono presenti le principali funzionalità disponibili e altre 3 viste che richiedono l'inserimento dei dati.\\
La seguente lista riassume quando detto:

\begin{itemize}
    \item \textbf{Linee di codice:} 3216;
    \item \textbf{Numero di classi:} 52;
    \item \textbf{Numero di file di configurazione:} 2;
    \item \textbf{Diagrammi dei casi d'uso:} 7;
    \item \textbf{Diagrammi delle classi:} 6;
    \item \textbf{Numero di viste:} 4;
    \item \textbf{Documenti realizzati:} 2.
\end{itemize}