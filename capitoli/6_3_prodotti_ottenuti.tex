% !TEX encoding = UTF-8
% !TEX TS-program = pdflatex
% !TEX root = ../tesi.tex
%**************************************************************

\section{Prodotti realizzati}\label{sec:prodotti-ottenuti}
Lo stage ha portato alla realizzazione del tool APAT che permette di analizzare il codice sorgente di un'applicazione e individuare eventuali problemi di sicurezza.
Il tool, scritto in Java, è composto da 3216 righe di codice, suddivise in 52 file, e dipende da due file di configurazione in formato JSON.

Per la progettazione del tool sono stati realizzati sei diagrammi delle classi e dei package costituendo di conseguenza il documento \textit{Manuale dello sviluppatore}.
Per maggiori dettagli consultare la sezione \hyperref[sec:progettazione]{\S4.4}.

Per l'analisi dei requisiti sono stati realizzati sette diagrammi dei casi d'uso componendo così il documento \textit{Analisi dei requisiti}.
Per maggiori dettagli consultare la sezione \hyperref[ch:analisi-requisiti]{\S3}.\\
Lo strumento APAT è composto da quattro viste, quella principale in cui sono presenti le principali funzionalità disponibili e altre tre viste che richiedono l'inserimento dei dati.\\
La seguente lista riassume quanto detto:

\begin{itemize}
    \item \textbf{Linee di codice:} 3216;
    \item \textbf{Numero di classi:} 52;
    \item \textbf{Numero di file di configurazione:} 2;
    \item \textbf{Diagrammi dei casi d'uso:} 7;
    \item \textbf{Diagrammi delle classi:} 6;
    \item \textbf{Numero di viste:} 4;
    \item \textbf{Documenti realizzati:} 2.
\end{itemize}