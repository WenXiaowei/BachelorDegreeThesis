
%**************************************************************
% Acronimi
%**************************************************************
\renewcommand{\acronymname}{Acronimi e abbreviazioni}

\newacronym[description={\glslink{apig}{Application Program Interface}}]
    {api}{API}{Application Program Interface}

\newacronym[description={\glslink{umlg}{Unified Modeling Language}}]
    {uml}{UML}{Unified Modeling Language}

\newacronym[description={\glslink{APK}{Android PacKage}}]
    {apk}{APK}{Android PacKage}

\newacronym[description={\glslink{dexg}{Dalvik Executable}}]
    {dex}{DEX}{Dalvik Executable}


%
%\newacronym[description={\glslink{}{}}]
%    {}{}{}
%
%\newacronym[description={\glslink{}{}}]
%    {}{}{}

%**************************************************************
% Glossario
%**************************************************************
%\renewcommand{\glossaryname}{Glossario}

\newglossaryentry{apig}
{
    name=\glslink{api}{API},
    text=Application Program Interface,
    sort=api,
    description={in informatica con il termine \emph{Application Programming Interface API} (ing. interfaccia di programmazione di un'applicazione) si indica ogni insieme di procedure disponibili al programmatore, di solito raggruppate a formare un set di strumenti specifici per l'espletamento di un determinato compito all'interno di un certo programma. La finalità è ottenere un'astrazione, di solito tra l'hardware e il programmatore o tra software a basso e quello ad alto livello semplificando così il lavoro di programmazione}
}

\newglossaryentry{umlg}
{
    name=\glslink{uml}{UML},
    text=UML,
    sort=uml,
    description={in ingegneria del software \emph{UML, Unified Modeling Language} (ing. linguaggio di modellazione unificato) è un linguaggio di modellazione e specifica basato sul paradigma object-oriented. L'\emph{UML} svolge un'importantissima funzione di ``lingua franca'' nella comunità della progettazione e programmazione a oggetti. Gran parte della letteratura di settore usa tale linguaggio per descrivere soluzioni analitiche e progettuali in modo sintetico e comprensibile a un vasto pubblico}
}

\newglossaryentry{repackaging}
{
    name=\glslink{repackaging}{repackaging},
    text=repackaging,
    sort=repackaging,
    description={in ingegneria del software, il termine repackaging indica il processo di creazione del pacchetto d'installazione a seguito di un processo di trasformazione contraria}
}

\newglossaryentry{dexg}
{
    name=\glslink{dex}{dex},
    text=DEX,
    sort=dex,
    description={File eseguibili Dalvik sono file di sviluppo apposte con il .dex estensione, e questi file DEX vengono utilizzati per inizializzare ed eseguire applicazioni sviluppate per il sistema operativo mobile Android.
    I dati memorizzati in questi file DEX include compilato il codice che individua e inizializza altri file di programma dell'applicazione associata necessario per eseguire il programma.}
}
\newglossaryentry{APK}
{
    name=\glslink{apk}{APK},
    text=APK,
    sort=apk,
    description={L'estensione APK indica un file Android Package. Questo formato di file, una variante del formato .JAR, è utilizzato per la distribuzione e l'installazione di componenti in dotazione sulla piattaforma per dispositivi mobili Android}
}

\newglossaryentry{cap}
{
    name=\glslink{cap}{cap},
    text=cap,
    sort=cap,
    description={I  file con estensione .CAP contengono dati acquisiti dai programmi di tracciamento del pacchetto di trasmissione dati. I dati sono in forma grezza, il che significa esattamente la stessa forma in cui è stato catturato.
    I file sono spesso definiti come file di traccia o file ossei. Salvare i pacchetti catturati usando il formato CAP può essere fatto da molti tipi di app, chiamati sniffer. Questi programmi quindi analizzano i dati acquisiti come richiesto dall'utente, applicando la filtrazione e l'elaborazione appropriate.}
}
%\newglossaryentry{}
%{
%    name=\glslink{}{},
%    text=,
%    sort=,
%    description={}
%}

